\phantomsection
\section*{8 相抵标准形}
\addcontentsline{toc}{section}{8 相抵标准形}

\vspace{2ex}

\centerline{\heiti A组}
\begin{enumerate}
    \item 首先取 $ \mathbf{R}^4 $ 的自然基 $\varepsilon_1, \varepsilon_2, \varepsilon_3, \varepsilon_4$,那么
          \[ (\alpha_1, \alpha_2, \alpha_3, \alpha_4) = (\varepsilon_1, \varepsilon_2, \varepsilon_3, \varepsilon_4) A,\enspace (\beta_1, \beta_2, \beta_3, \beta_4) = (\varepsilon_1, \varepsilon_2, \varepsilon_3, \varepsilon_4) B \]
          其中
          \[ A = \begin{pmatrix}
                  1 & 1  & 1  & 1  \\
                  1 & 1  & -1 & -1 \\
                  1 & -1 & 1  & -1 \\
                  1 & -1 & -1 & 1
              \end{pmatrix},\enspace B = \begin{pmatrix}
                  1 & 2 & 1 & 0  \\
                  1 & 1 & 1 & 1  \\
                  0 & 3 & 0 & -1 \\
                  1 & 1 & 0 & -1
              \end{pmatrix} \]
          由此可知
          \[ (\beta_1, \beta_2, \beta_3, \beta_4) = (\varepsilon_1, \varepsilon_2, \varepsilon_3, \varepsilon_4) B = (\alpha_1, \alpha_2, \alpha_3, \alpha_4) A^{-1} B \]
          所以由基 $ \alpha_1, \alpha_2, \alpha_3, \alpha_4 $ 到基 $ \beta_1, \beta_2, \beta_3, \beta_4 $ 的过渡矩阵为
          \[ A^{-1} B = \frac{1}{4} \begin{pmatrix}
                  1 & 1  & 1  & 1  \\
                  1 & 1  & -1 & -1 \\
                  1 & -1 & 1  & -1 \\
                  1 & -1 & -1 & 1
              \end{pmatrix} \begin{pmatrix}
                  1 & 2 & 1 & 0  \\
                  1 & 1 & 1 & 1  \\
                  0 & 3 & 0 & -1 \\
                  1 & 1 & 0 & -1
              \end{pmatrix} = \frac{1}{4} \begin{pmatrix}
                  3  & 7  & 2 & -1 \\
                  1  & -1 & 2 & 3  \\
                  -1 & 3  & 0 & -1 \\
                  1  & -1 & 0 & -1
              \end{pmatrix} \]
          另外,由于 $ \xi $ 在基 $ \varepsilon_1, \varepsilon_2, \varepsilon_3, \varepsilon_4 $ 下的坐标为 $ X_0 = (1, 0, 0, -1)^{\mathrm{T}} $,所以在 $ \alpha_1, \alpha_2, \alpha_3, \alpha_4 $ 下的坐标为
          \[ A^{-1} X_0 = \frac{1}{4} \begin{pmatrix}
                  1 & 1  & 1  & 1  \\
                  1 & 1  & -1 & -1 \\
                  1 & -1 & 1  & -1 \\
                  1 & -1 & -1 & 1
              \end{pmatrix} \begin{pmatrix}
                  1 \\ 0 \\ 0 \\ -1
              \end{pmatrix} = \begin{pmatrix}
                  0 \\ \frac{1}{2} \\ \frac{1}{2} \\ 0
              \end{pmatrix} \]

    \item 证明:考虑矩阵 $A$ 的行向量组的极大线性无关组,若添加的一行可由其极大线性无关组线性表示,则秩不变. 否则秩增加 $1$.

    \item 证明:设 $A$ 的行向量组为 $\alpha_1,\alpha_2,\ldots,\alpha_s$,$r(A)=r$; $B$ 的行向量组为 $\alpha_1,\alpha_2,\ldots,\alpha_m ,r(B)=k$.
          不妨设:$B$ 的行向量组的极大线性无关组为 $\alpha_1,\alpha_2,\ldots,\alpha_k,\alpha_{i_1},\ldots,\alpha_{i_{r-k}}$,其中 $\alpha_{i_1},\ldots,\alpha_{i_{r-k}}$(共 $r-k$ 个向量)是包含在 $\alpha_{m+1},\ldots,\alpha_s$(共 $s-m$ 个向量)之中的. 显然有 $r-k \leqslant s-m $,即
          \[r(B)=k\geqslant r+m-s=r(A)+m-s.\]
\end{enumerate}

\centerline{\heiti B组}
\begin{enumerate}
    \item 已知 $r(A+B) \leqslant r(A)+r(B)$,把 $B$ 写成 $-B$ 则有 $r(A-B) \leqslant r(A)+r(-B)=r(A)+r(B)$. 不等式右半部分得证.

          另外,$r(A)=r(A-B+B) \leqslant r(A-B)+r(B)$,从而 $r(A-B) \geqslant r(A)-r(B)$,当然,加个绝对值也是没有问题的:$r(A-B) \geqslant \lvert r(A)-r(B) \rvert$. 同理,有 $r(A+B) \geqslant \lvert r(A)+r(B) \rvert$. 证毕.

    \item $V$ 的基 $B_1$ 到 $B_2$ 的过渡矩阵 $P$ 具有下述形式:
          \[P=\begin{pmatrix}\im & B_1 \\ 0 & B_2\end{pmatrix}\]
          其中 $B_1,B_2$ 分别是域 $\mathbf{F}$ 上 $m\times (n-m),(n-m)\times (n-m)$ 矩阵,
          \[\beta_j=b_{j1}\delta_1+\cdots+b_{jm}\delta_m+b_{j,m+1}\delta_{m+1}+\cdots+b_{jn}a_n\]
          其中 $j=m+1,\ldots,n$. 于是
          \[\beta_j+W=b_{j,m+1}(\alpha_{m+1}+W)+\cdots+b_{jn}(\alpha_n+W)\]
          因此商空间 $V/W$ 的基 $\alpha_{m+1}+W,\ldots,\alpha_n+W$ 到 $\beta_{m+1}+W,\ldots,\beta_n+W$ 的过渡矩阵是 $B_2$.

    \item 证明:我们设 $ \beta_1 = \alpha_1 + \alpha_2,\enspace \ldots, \beta_{n - 1} = \alpha_{n - 1} + \alpha_n,\enspace \beta_n = \alpha_n + \alpha_1 $. 由于
          \[ (\beta_1, \beta_2, \ldots, \beta_n) = (\alpha_1, \alpha_2, \ldots, \alpha_n) A \]
          其中
          \[ A = \begin{pmatrix}
                  1 &   &        &   & 1 \\
                  1 & 1 &        &   &   \\
                    & 1 & \ddots &   &   \\
                    &   & \ddots & 1 &   \\
                    &   &        & 1 & 1
              \end{pmatrix} \]
          满足 $ |A| = 1 + (-1)^{n + 1} = 2 $,所以 $ A $ 可逆. 根据教材命题 3.10.3 可得 $ \alpha_1, \alpha_2, \ldots, \alpha_n $ 与 $ \beta_1, \beta_2, \ldots, \beta_n $ 等价. 当然,这说明 $ \alpha_1, \alpha_2, \ldots, \alpha_n $ 线性无关的充要条件是 $ \beta_1, \beta_2, \ldots, \beta_n $ 线性无关.

    \item 记 $ B_1 = \{\vec{e}_{11}, \vec{e}_{12}, \vec{e}_{21}, \vec{e}_{22}\} $,$ B_2 = \{\vec{g}_1, \vec{g}_2, \vec{g}_3, \vec{g}_4\} $.
          \begin{enumerate}
              \item 设 $ k_1 \vec{g}_1 + k_2 \vec{g}_2 + k_3 \vec{g}_3 + k_4 \vec{g}_4 = O $,即
                    \[ \begin{pmatrix} k_1 + k_2 + k_3 + k_4 & k_2 + k_3 + k_4 \\ k_3 + k_4 & k_4 \end{pmatrix} = \begin{pmatrix} 0 & 0 \\ 0 & 0 \end{pmatrix} \]
                    这个方程对应的四元齐次线性方程组的解为
                    \[ k_1 = k_2 = k_3 = k_4 = 0 \]
                    所以 $ \vec{g}_1, \vec{g}_2, \vec{g}_3, \vec{g}_4 $ 线性无关,从而是 $ \mathbf{M}_2(\mathbf{R}) $ 的一组基.

              \item \label{item:11:B:4:2}
                    $ \mathbf{M}_2(\mathbf{R}) \cong \mathbf{R}^4 $,所以 $ \vec{e}_{11}, \vec{e}_{12}, \vec{e}_{21}, \vec{e}_{22} $ 可表示为 $ \mathbf{R}^4 $ 的自然基 $ \vec{e}_1, \vec{e}_2, \vec{e}_3, \vec{e}_4 $,而 $ \vec{g}_1, \vec{g}_2, \vec{g}_3, \vec{g}_4 $ 可表示为 $ (1, 0, 0, 0)^{\mathrm{T}},\allowbreak (1, 1, 0, 0)^{\mathrm{T}},\allowbreak (1, 1, 1, 0)^{\mathrm{T}},\allowbreak (1, 1, 1, 1)^{\mathrm{T}} $. 于是由
                    \[ (\vec{g}_1, \vec{g}_2, \vec{g}_3, \vec{g}_4) = (\vec{e}_{11}, \vec{e}_{12}, \vec{e}_{21}, \vec{e}_{22}) C \]
                    可得
                    \[ (\vec{e}_{11}, \vec{e}_{12}, \vec{e}_{21}, \vec{e}_{22}) = (\vec{g}_1, \vec{g}_2, \vec{g}_3, \vec{g}_4) C^{-1} \]
                    其中
                    \[ C = \begin{pmatrix}
                            1 & 1 & 1 & 1 \\
                            0 & 1 & 1 & 1 \\
                            0 & 0 & 1 & 1 \\
                            0 & 0 & 0 & 1
                        \end{pmatrix},\enspace C^{-1} = \begin{pmatrix}
                            1 & -1 & 0  & 0  \\
                            0 & 1  & -1 & 0  \\
                            0 & 0  & 1  & -1 \\
                            0 & 0  & 0  & 1
                        \end{pmatrix} \]
                    所以基 $ B_2 $ 变换为基 $ B_1 $ 的变换矩阵为 $ C^{-1} $.

              \item 在 $ A^2 = A $ 中选取较为简单的. 例如,由
                    \[ \begin{pmatrix} a & b \\ 0 & 0 \end{pmatrix}^2 = \begin{pmatrix} a^2 & ab \\ 0 & 0 \end{pmatrix} = \begin{pmatrix} a & b \\ 0 & 0 \end{pmatrix} \]
                    取 $ a = 1 $,$ b = 0 $ 或 $ b = 1 $ 得
                    \[ A_1 = \begin{pmatrix} 1 & 0 \\ 0 & 0 \end{pmatrix},\enspace A_2 = \begin{pmatrix} 1 & 1 \\ 0 & 0 \end{pmatrix} \]
                    由
                    \[ \begin{pmatrix} 0 & 0 \\ c & d \end{pmatrix}^2 = \begin{pmatrix} 0 & 0 \\ cd & d^2 \end{pmatrix} = \begin{pmatrix} 0 & 0 \\ c & d \end{pmatrix} \]
                    取 $ d = 1 $,$ c = 0 $ 或 $ c = 1 $ 得
                    \[ A_3 = \begin{pmatrix} 0 & 0 \\ 0 & 1 \end{pmatrix},\enspace A_4 = \begin{pmatrix} 0 & 0 \\ 1 & 1 \end{pmatrix} \]
                    上面的 $ A_1, A_2, A_3, A_4 $ 均满足 $ A_i^2 = A_i \enspace(i = 1, \ldots, 4) $,而且线性无关. 所以它们是 $ \mathbf{M}_2(\mathbf{R}) $ 的一组基 $ B_3 $.

              \item 先求基 $ B_2 $ 变换为基 $ B_3 $ 的变换矩阵 $ D $,即
                    \[ (A_1, A_2, A_3, A_4) = (\vec{g}_1, \vec{g}_2, \vec{g}_3, \vec{g}_4) D \]
                    由 \ref*{item:11:B:4:2} 所述,此时有
                    \[ \begin{pmatrix}
                            1 & 1 & 0 & 0 \\
                            0 & 1 & 0 & 0 \\
                            0 & 0 & 0 & 1 \\
                            0 & 0 & 1 & 1
                        \end{pmatrix} = CD \]
                    所以
                    \[ D = C^{-1} \begin{pmatrix}
                            1 & 1 & 0 & 0 \\
                            0 & 1 & 0 & 0 \\
                            0 & 0 & 0 & 1 \\
                            0 & 0 & 1 & 1
                        \end{pmatrix} = \begin{pmatrix}
                            1 & 0 & 0  & 0  \\
                            0 & 1 & 0  & -1 \\
                            0 & 0 & -1 & 0  \\
                            0 & 0 & 1  & 1
                        \end{pmatrix} \]
                    由于矩阵 $ A $ 关于基 $ B_2 $ 的坐标为 $ X = (1, 1, 1, 1)^{\mathrm{T}} $,所以 $ A $ 关于基 $ B_3 $ 的坐标为
                    \[ Y = D^{-1} X = \begin{pmatrix}
                            1 & 0 & 0  & 0  \\
                            0 & 1 & 0  & -1 \\
                            0 & 0 & -1 & 0  \\
                            0 & 0 & 1  & 1
                        \end{pmatrix}^{-1} \begin{pmatrix} 1 \\ 1 \\ 1 \\ 1 \end{pmatrix} = \begin{pmatrix}
                            1 & 0 & 0  & 0 \\
                            0 & 1 & 1  & 1 \\
                            0 & 0 & -1 & 0 \\
                            0 & 0 & 1  & 1
                        \end{pmatrix}^{-1} \begin{pmatrix} 1 \\ 1 \\ 1 \\ 1 \end{pmatrix} = \begin{pmatrix} 1 \\ 3 \\ -1 \\ 2 \end{pmatrix} \]
          \end{enumerate}

    \item \begin{enumerate}
              \item 初等变换即可.

              \item 同上.

              \item 矩阵 $A$ 秩为 $r$ 可写作 $A=P\begin{pmatrix}E_r & 0 \\ 0 & 0\end{pmatrix}Q = P(E_{11}+E_{22}+\cdots+E_{rr})Q$($E_r$ 是 $r\times r$ 的单位矩阵,$E_{ii}$ 是 $n\times n$ 的只有第 $i$ 行 $i$ 列的这个元素为 1,其他元素均为 0 的矩阵). 每个 $PE_{ii}Q$ 都是秩为 1 的矩阵,故得证.

              \item 记 $r(A)=r$,把 $A$ 写成 $P\begin{pmatrix}E_r & 0 \\ 0 & 0\end{pmatrix}Q$ 的形式. 构造 $B=Q^{-1}\begin{pmatrix}E_r & 0 \\ 0 & 0\end{pmatrix}P^{-1}$ 可以发现其满足条件,故得证.
          \end{enumerate}

    \item $r(BC)\leqslant r(B) \leqslant 1$,得证.

          反之,若 $A$ 是秩为 1 的 $3\times 3$ 矩阵,则存在可逆矩阵 $P,Q$ 使得 $A=P^{-1}E_{11}Q^{-1}$,其中 $E_{11}=\begin{pmatrix}1 & 0 & 0 \\ 0 & 0 & 0 \\ 0 & 0 & 0\end{pmatrix}=\begin{pmatrix}1 \\ 0 \\ 0\end{pmatrix}\begin{pmatrix}1 & 0 & 0\end{pmatrix}$. 则取 $B=P^{-1}\begin{pmatrix}1 \\ 0 \\ 0\end{pmatrix},C=\begin{pmatrix}1 & 0 & 0\end{pmatrix}Q^{-1}$,有 $A=BC$,证毕.

    \item \begin{enumerate}
              \item $r(\alpha \alpha^{\mathrm{T}})\leqslant r(\alpha) = 1,r(\beta \beta^{\mathrm{T}})\leqslant r(\beta) = 1$. 由 $r(A+B) \leqslant r(A)+r(B)$ 得 $r(A)=r(\alpha \alpha^{\mathrm{T}}+\beta \beta^{\mathrm{T}}) \leqslant r(\alpha)+r(\beta)=2$.

              \item 若 $\alpha,\beta$ 均为 $\vec{0}$,显然. 否则假设 $\alpha$ 不为 $\vec{0}$,则由于两向量线性相关,必有确定的 $k$ 使得 $\beta = k\alpha$,把 $\beta$ 用 $\alpha$ 表示之后易证.
          \end{enumerate}

    \item $r(A)=r$ 则 $AX=0$ 的解空间维数 $\dim N(A) = n-r$. 由 $r(A)+r(B)=k$ 得 $r(B)=k-r \leqslant n-r=\dim N(A)$. 要求 $AB=O$,说明 $B$ 的列向量均为 $AX=0$ 的解,那么只需要选择合适的列向量组拼接成 $B$ 即可(这一定能做到,因为 $B$ 维数不会超过解空间维数).

    \item 由于 $A$ 是 $m\times n$ 矩阵,$r(A)=m$,可知对于矩阵 $A$ 做初等列变换,可使其前 $m$ 列变为单位矩阵,后 $n-m$ 列变为全 0 列. 因此,存在 $n$ 阶可逆矩阵 $P$ 使得
          \[AP=\begin{pmatrix}E_m & O_{m\times (n-m)}\end{pmatrix}\]
          于是\[AP(AP)^{\mathrm{T}} = \begin{pmatrix}E & O\end{pmatrix} \begin{pmatrix}E \\ O\end{pmatrix}=E_m\]
          所以存在 $B=(PP^{\mathrm{T}}A^{\mathrm{T}})$ 为 $n\times m$ 矩阵,使 $AB=E$.

    \item 利用 $A,B$ 的相抵标准形. 存在 $n$ 阶可逆矩阵 $P_1,Q_1,P_2,Q_2$ 使得
          \[P_1AQ_1=\begin{pmatrix}E_{r_A} & O \\ O & O\end{pmatrix},P_2BQ_2=\begin{pmatrix}O & O \\ O & E_{r_B}\end{pmatrix}\]
          于是 \[AQ_1=P_1^{-1}\begin{pmatrix}E_{r_A} & O \\ O & O\end{pmatrix},P_2B=\begin{pmatrix}O & O \\ O & E_{r_B}\end{pmatrix}Q_2^{-1}\]
          所以 \[AQ_1P_2B=P_1^{-1}\begin{pmatrix}E_{r_A} & O \\ O & O\end{pmatrix}\begin{pmatrix}O & O \\ O & E_{r_B}\end{pmatrix}Q_2^{-1}=O\]
          取 $M=Q_1P_2$ 即可.

    \item \begin{enumerate}
              \item 易证,此处略去.

              \item 注意到 $B$ 的列向量均为 $AX=0$ 的解,设 $AX=0$ 的基础解系为 $\alpha_1,\ldots,\alpha_t(t=n-r)$,则易知
                    \[B_{11}=(\alpha_1,0,\ldots,0),B_{12}=(0,\alpha_1,\ldots,0),\ldots,B_{1n}=(0,0,\ldots,\alpha_1),\]
                    \[B_{21}=(\alpha_2,0,\ldots,0),B_{22}=(0,\alpha_2,\ldots,0),\ldots,B_{2n}=(0,0,\ldots,\alpha_2),\]
                    \[\vdots\]
                    \[B_{t1}=(\alpha_t,0,\ldots,0),B_{t2}=(0,\alpha_t,\ldots,0),\ldots,B_{tn}=(0,0,\ldots,\alpha_t)\]
                    为 $S(A)$ 的一组基,故 $\dim S(A)=n(n-r)$.
          \end{enumerate}
\end{enumerate}

\centerline{\heiti C组}
\begin{enumerate}
    \item 对 $\begin{pmatrix}E_n & A' \\ A & E_s\end{pmatrix}$ 利用打洞原理有
          \[\begin{pmatrix}E_n-A'A & O \\ O & E_s\end{pmatrix} \leftarrow \begin{pmatrix}E_n & A' \\ A & E_s\end{pmatrix} \rightarrow \begin{pmatrix}E_n & O \\ O & E_s-AA'\end{pmatrix}\]
          所以 $r\begin{pmatrix}E_n-A'A & O \\ O & E_s\end{pmatrix}=r\begin{pmatrix}E_n & O \\ O & E_s-AA'\end{pmatrix}$,即 $s+r(E_n-A'A)=n+r(E_s-AA')$,即
          \[r(E_n-A'A)-r(E_s-AA')=n-s.\]

    \item \begin{enumerate}
              \item 由 \[\begin{pmatrix}A & 0 \\ 0 & B\end{pmatrix}\rightarrow \begin{pmatrix}A & AC \\ 0 & B\end{pmatrix}\rightarrow \begin{pmatrix}A & AC+BD \\ 0 & B\end{pmatrix}=\begin{pmatrix}A & E \\ 0 & B\end{pmatrix}\]
                    \[\rightarrow \begin{pmatrix}0 & E \\ -AB & B\end{pmatrix}\rightarrow \begin{pmatrix}0 & E \\ AB & 0\end{pmatrix}\]
                    可得.

              \item 用分块矩阵的方法,我们知道
                    \[\begin{pmatrix}A & O \\ O & B\end{pmatrix}\rightarrow \begin{pmatrix}A & O \\ A & B\end{pmatrix}\rightarrow \begin{pmatrix}A & A \\ A & A+B\end{pmatrix}\]
                    结合 $AB=BA$,我们知道
                    \[\begin{pmatrix}A & A \\ A & A+B\end{pmatrix}\begin{pmatrix}A+B & O \\ -A & E\end{pmatrix}=\begin{pmatrix}AB & A \\ O & A+B\end{pmatrix}\]
                    于是
                    \[r(A)+r(B)=r\begin{pmatrix}A & O \\ O & B\end{pmatrix}=r\begin{pmatrix}A & A \\ A & A+B\end{pmatrix}\geqslant \begin{pmatrix}AB & A \\ O & A+B\end{pmatrix}\geqslant r(AB)+r(A+B)\]
          \end{enumerate}

    \item 略有超纲,使用裴蜀定理. $\exists u(x),v(x),u(x)f_1(x)+v(x)f_2(x)=1 $,于是
          \begin{align*}
              r\begin{pmatrix}f_1(A) & O \\ O & f_2(A)\end{pmatrix} & = r\begin{pmatrix}f_1(A) & f_1(A)u(A)+f_2(A)v(A) \\ O & f_2(A)\end{pmatrix}=r\begin{pmatrix}f_1(A) & E \\ O & f_2(A)\end{pmatrix} \\
                                                                    & =r\begin{pmatrix}f_1(A) & E \\ -f_2(A)f_1(A) & O\end{pmatrix}=r\begin{pmatrix}O & E \\ f(A) & O\end{pmatrix}
          \end{align*}

    \item 由于 $A$ 是列满秩矩阵,$B$ 是行满秩矩阵,知存在可逆矩阵 $P_{3\times 3},Q_{2\times 2}$ 使得
          \[A=P\begin{pmatrix}E_2 \\ O\end{pmatrix},B=\begin{pmatrix}E_2 & O\end{pmatrix}Q\]
          于是
          \[BA=\begin{pmatrix}E_2 & O\end{pmatrix}QP\begin{pmatrix}E_2 \\ O\end{pmatrix}\]
          由 $(AB)^2=9AB$ 有 \[P\begin{pmatrix}E_2 \\ O\end{pmatrix}\begin{pmatrix}E_2 & O\end{pmatrix}QP\begin{pmatrix}E_2 \\ O\end{pmatrix}\begin{pmatrix}E_2 & O\end{pmatrix}Q=9P\begin{pmatrix}E_2 \\ O\end{pmatrix}\begin{pmatrix}E_2 & O\end{pmatrix}Q\]
          即
          \[\begin{pmatrix}E_2 \\ O\end{pmatrix}BA\begin{pmatrix}E_2 & O\end{pmatrix}=9\begin{pmatrix}E_2 \\ O\end{pmatrix}\begin{pmatrix}E_2 & O\end{pmatrix}\]
          也就是 \[\begin{pmatrix}BA & O \\ O & O\end{pmatrix}=\begin{pmatrix}9E_2 & 0 \\ 0 & 0\end{pmatrix}\]
          所以 $BA=9E_2$.

    \item 本题直接求核空间困难,但由于我们只需求维数,所以我们转而求像空间维数并通过维数公式求核空间维数.

          取 $ \mathbf{F}^{n \times p} $ 的自然基 $ \vec{e}_{ij} = (e_{kl})_{n \times p} $,其中 $ e_{kl} = \delta_{ik} \delta_{jl} $,即第 $ i $ 行第 $ j $ 列元素为 $ 1 $,其余元素为 $ 0 $. 则 $ \im \sigma = \spa(\sigma(\vec{e}_{11}), \ldots, \sigma(\vec{e}_{np})) $.

          取 $ A $ 的列向量组 $ A = (\alpha_1, \alpha_2, \ldots, \alpha_n) $,则可将 $ \sigma(\vec{e}_{ij}) $ 排列如下:
          \[ \begin{matrix}
                  (a_1, 0, \ldots, 0) & (0, a_1, \ldots, 0) & \cdots & (0, 0, \ldots, a_1) \\
                  (a_2, 0, \ldots, 0) & (0, a_2, \ldots, 0) & \cdots & (0, 0, \ldots, a_2) \\
                  \vdots              & \vdots              & \ddots & \vdots              \\
                  (a_n, 0, \ldots, 0) & (0, a_n, \ldots, 0) & \cdots & (0, 0, \ldots, a_n)
              \end{matrix} \]
          记 $ r(A) = r $,则 $ \alpha_1, \alpha_2, \ldots, \alpha_n $ 的极大线性无关组有 $ r $ 个向量,不妨设为 $ \alpha_1, \alpha_2, \ldots, \alpha_r $. 那么下列向量:
          \[ \begin{matrix}
                  (a_{r+1}, 0, \ldots, 0) & (0, a_{r+1}, \ldots, 0) & \cdots & (0, 0, \ldots, a_{r+1}) \\
                  \vdots                  & \vdots                  & \ddots & \vdots                  \\
                  (a_n, 0, \ldots, 0)     & (0, a_n, \ldots, 0)     & \cdots & (0, 0, \ldots, a_n)
              \end{matrix} \]

          均可以被其他向量线性表出. 观察除了上述向量的剩下的向量,可以发现这 $ rp $ 个向量线性无关,它们是 $ \im \sigma $ 的一组基,因此 $ \dim \im \sigma = rp $. 由维数公式,$ \dim \ker \sigma = \dim \mathbf{F}^{n \times p} - \dim \im \sigma = (n - r)p $.
\end{enumerate}

\clearpage
