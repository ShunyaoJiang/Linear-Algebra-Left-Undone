\chapter{相似标准形:动机与基础}

回顾线性代数的一大目标,我们希望找到出发空间和到达空间合适的基使得线性映射在这两组基下的表示更简单. 我们已经在相抵标准形一节中使用线性映射基本定理给出了一种可行的构造. 从本讲起,我们把眼光放在线性变换,即\textbf{出发空间和到达空间相同的线性映射,并且我们关注出发空间与到达空间取同一组基的时候,如何得到更简单的标准形}. 这一矩阵标准形我们称之为相似标准形,这也是我们讨论的第一个核心概念.

当然我们会发现关于相似标准形讨论的限制显然比之前相抵标准形的更大. 回顾\autoref{thm:相抵标准形} 中取得相抵标准形对应的基的方法,我们显然无法保证出发空间和到达空间取到的是同一组基,即使是在出发空间与到达空间一致的情况下. 因此本节我们将会探索另一种解决这一线性代数核心问题的范式,这就引出了接下来数讲的讨论核心——不变子空间.

\section{相似的定义与性质}

为了讨论相似标准形,我们需要首先从相似这一概念谈起. 正如我们在相抵标准形中的讨论那样,我们会综合线性映射和矩阵两个观点来讨论这一问题. 从线性映射的角度出发,我们的想法是对一个$n$维线性空间$V$上的线性变换$\sigma\in\mathcal{L}(V)$,找到一组基$\{\alpha_1,\ldots,\alpha_n\}$,使得$T$在这组基下的矩阵``比较简单'',即我们有
\[\sigma(\alpha_1,\ldots,\alpha_n)=(\alpha_1,\ldots,\alpha_n)A,\]
我们希望$A$是``比较简单''的,就像相抵标准形那样,有尽可能多的$0$,非零元素的排列规律也尽可能简单.

如果从矩阵的角度出发,很自然地,我们会希望与上面映射的目标建立联系,我们会想起任意矩阵$A$都是线性变换$\sigma(\alpha)=A\alpha$在自然基下的表示矩阵. 此时我们的目标可以与映射的目标产生联系,即我们希望找到另一组基,使得$\sigma$在这组基下的矩阵$B$比较简单. 此时$A$和$B$的关系是同一个线性变换$\sigma$在不同基下的表示矩阵,事实上这一关系我们其实并不陌生,早在相抵标准形一讲中我们就提到如下定理:
\begin{theorem}{基的选择对变换矩阵的影响}{}
    设线性变换$\sigma \in \mathcal{L}(V,V)$,$B_1=\{\alpha_1,\ldots,\alpha_n\}$和$B_2=\{\beta_1,\ldots,\beta_n\}$是线性空间的$V(\mathbf{F})$的两组基,基$B_1$变为基$B_2$的变换矩阵为$C$,如果$\sigma$在基$B_1$下的矩阵为$A$,则$\sigma$关于基$B_2$所对应的矩阵为$C^{-1}AC$.
\end{theorem}
这一定理研究了同一个线性变换在不同基下表示矩阵之间的关系,契合了我们的需求. 我们可以将上述同一线性变换在不同基下的矩阵表示之间的关系称为\term{相似}. 当然这是从线性映射的角度出发的定义,有时我们谈到相似可能仅仅是矩阵层面的关系,因此我们给出直接的使用矩阵定义如下:
\begin{definition}{}{}
    若对于$A,B\in \mathbf{M}_n(\mathbf{F})$,存在可逆矩阵$C\in \mathbf{M}_n(\mathbf{F})$, 使得$C^{-1}AC=B$,则称$A$相似于$B$,记作$A\sim B$.
\end{definition}

因此从矩阵的角度出发,我们的目标是找到一个可逆矩阵$P$,使得$P^{-1}AP$是``比较简单''的. 或者说,我们的目标是找到一个与$A$相似的``比较简单''的矩阵$B$,这样合乎要求的矩阵$B$就是我们所谓的矩阵$A$的\term{相似标准形}.

接下来我们讨论相似的一些基本性质:
\begin{enumerate}
    \item 相似是一种等价关系;两矩阵相似必相抵(相似矩阵的秩相等);

          \begin{proof}
              等价关系的证明:

              自反性:对于$\forall A \in \mathbf{M}_n(\mathbf{F})$,有$E^{-1}AE=A$,从而$A \sim A$.

              对称性:对于$A,B\in \mathbf{M}_n(\mathbf{F})$,若$A\sim B$,则存在可逆矩阵$C \in \mathbf{M}_n(\mathbf{F})$,
              使得$C^{-1}AC=B$,从而$(C^{-1})^{-1}B(C^{-1})=A,B \sim A$.

              传递性:对于$A,B,C \in \mathbf{M}_n(\mathbf{F})$,若$A\sim B,B\sim C$,存在可逆矩阵$P,Q \in \mathbf{M}_n(\mathbf{F})$,使得$P^{-1}AP=B,Q^{-1}BQ=C$,从而$PQ$可逆,$(PQ)^{-1}A(PQ)=C,A\sim C$.
              此外,由于乘以可逆矩阵不改变原矩阵的秩,因此$r(P^{-1}AP)=r(A)$,即两矩阵相似必定秩相等,因此也必定相抵.
          \end{proof}

          需要说明的是相似必定相抵但反之不一定成立,相抵的两个矩阵甚至可能不是方阵,但相似有这一要求. 即使是方阵也不一定成立,在习题中我们会要求读者给出例子.

    \item $A\sim B$可以得到$A^\mathrm{T}\sim B^\mathrm{T}$,$A^m\sim B^m$. 更一般地,对于任意多项式$f(x)$都有$f(A)\sim f(B)$,且若$B=P^{-1}AP$,有$f(B)=P^{-1}f(A)P$. 除此之外还有$A,B$可逆时,$A^{-1}\sim B^{-1}$,$A^*\sim B^*$;

          \begin{proof}
              \begin{enumerate}
                  \item 由于$B^\mathrm{T}=(C^{-1}AC)^\mathrm{T}=C^\mathrm{T}A^\mathrm{T}(C^{-1})^\mathrm{T}=(C^\mathrm{T})^{-1}A^\mathrm{T}C^\mathrm{T}$,因此$A^\mathrm{T}\sim B^\mathrm{T}$;

                  \item 由于$B^m=(C^{-1}AC)^m=C^{-1}A^mC$(乘积展开后中间$C^{-1}C$可以消去),因此$A^m\sim B^m$;

                  \item 设$f(A)=a_0E+a_1A+\cdots+a_mA^m$,则$f(B)=a_0P^{-1}EP+a_1P^{-1}AP+\cdots+a_mP^{-1}A^mP=P^{-1}f(A)P$,因此$f(A)\sim f(B)$,且$f(B)=P^{-1}f(A)P$.

                  \item 由于$B^{-1}=(C^{-1}AC)^{-1}=C^{-1}A^{-1}C$,因此$A^{-1}\sim B^{-1}$,$A^*\sim B^*$.

                  \item 由于$B^*=(C^{-1}AC)^*=C^{-1}A^*C$,因此$A^*\sim B^*$.
              \end{enumerate}
          \end{proof}

    \item $A_1\sim B_1$,$A_2\sim B_2$不一定有$A_1+A_2\sim B_1+B_2$,只有当$P^{-1}A_1P=B_1,P^{-1}A_2P=B_2$时(即相同的过渡矩阵$P$)才有$P^{-1}(A_1+A_2)P=B_1+B_2$;

    \item 若$A_1\sim B_1$,$A_2\sim B_2$,则有
          \[ \begin{pmatrix}
                  A_1 & O \\ O & A_2
              \end{pmatrix}\sim\begin{pmatrix}
                  B_1 & O \\ O & B_2
              \end{pmatrix};\]

          \begin{proof}
              设$P_1^{-1}A_1P_1=B_1$,$P_2^{-1}A_2P_2=B_2$,则
              \[ \begin{pmatrix}
                      P_1 & O \\ O & P_2
                  \end{pmatrix}^{-1}\begin{pmatrix}
                      A_1 & O \\ O & A_2
                  \end{pmatrix}\begin{pmatrix}
                      P_1 & O \\ O & P_2
                  \end{pmatrix}=\begin{pmatrix}
                      P_1^{-1}A_1P_1 & O \\ O & P_2^{-1}A_2P_2
                  \end{pmatrix}=\begin{pmatrix}
                      B_1 & O \\ O & B_2
                  \end{pmatrix}.\]
          \end{proof}

    \item 与幂等矩阵相似的仍幂等,与对合矩阵相似的仍对合,与幂零矩阵相似的仍幂零(但与正交矩阵相似的不一定正交,但与正交矩阵正交相似的是正交矩阵).

          \begin{proof}
              我们一边回顾上面这些特殊矩阵的定义一边证明结论:
              \begin{enumerate}
                  \item 设$A$是幂等矩阵,则$A^2=A$,设$B=P^{-1}AP$,则$B^2=P^{-1}A^2P=P^{-1}AP=B$,因此$B$也是幂等矩阵,即与幂等矩阵相似的仍幂等;

                  \item 设$A$是对合矩阵,则$A^2=E$,设$B=P^{-1}AP$,则$B^2=P^{-1}A^2P=P^{-1}EP=P^{-1}P=E$,因此$B$也是对合矩阵,即与对合矩阵相似的仍对合;

                  \item 设$A$是幂零矩阵,则存在正整数$m$使得$A^m=O$,设$B=P^{-1}AP$,则$B^m=P^{-1}A^mP=P^{-1}OP=O$,因此$B$也是幂零矩阵,即与幂零矩阵相似的仍幂零;

                  \item 设$A$是正交矩阵,则$A^{-1}=A^\mathrm{T}$,设$B=P^{-1}AP$,则$B^{-1}=(P^{-1}AP)^{-1}=P^{-1}A^{-1}P=P^{-1}A^\mathrm{T}P$,只有当$P$是正交矩阵时,即$P^{-1}=P^\mathrm{T}$时,$B^{-1}=P^{-1}A^\mathrm{T}P=P^\mathrm{T}A^\mathrm{T}(P^{-1})^\mathrm{T}=(P^{-1}AP)^\mathrm{T}=B^\mathrm{T}$,因此与正交矩阵相似的不一定正交,只有过渡矩阵是正交矩阵时才一定正交.
              \end{enumerate}
          \end{proof}
\end{enumerate}

\section{不变子空间}

在介绍完相似的概念后,我们将正式开始我们的主线任务. 原谅我再次重复一遍,我们要考虑$n$维线性空间$V$上的线性变换$\sigma\in\mathcal{L}(V)$,我们希望找到一组基$\{\alpha_1,\ldots,\alpha_n\}$,使得$T$在这组基下的矩阵``比较简单'',即我们有
\[\sigma(\alpha_1,\ldots,\alpha_n)=(\alpha_1,\ldots,\alpha_n)A,\]
我们希望$A$是``比较简单''的,就像相抵标准形那样,有尽可能多的$0$,非零元素的排列规律也尽可能简单. 自然地,我们会希望$A$可以是对角矩阵,或者退而求其次至少是分块对角矩阵.

事实上,如果让我们直接思考如何获得这样的矩阵是非常容易使人迷茫的,但如果我们尝试从希望的矩阵的形式反推所取基需要的性质,应当能窥见一些端倪. 事实上,我们假设$A$为分块对角矩阵,即
\[A=\begin{pmatrix}
        A_1 &     &        &     \\
            & A_2 &        &     \\
            &     & \ddots &     \\
            &     &        & A_m
    \end{pmatrix},\]
设$A_i\enspace(i=1,\ldots,m)$的阶数为$r_i$. 我们进一步展开$A$的形式:
\[\begin{pmatrix}
        a_{11}   & \cdots & a_{1r_1}   &                   &        &                     &        &                     &        &               \\
        \vdots   & \ddots & \vdots     &                   &        &                     &        &                     &        &               \\
        a_{r_11} & \cdots & a_{r_1r_1} &                   &        &                     &        &                     &        &               \\
                 &        &            & a_{r_1+1,r_1+1}   & \cdots & a_{r_1+1,r_1+r_2}   &        &                     &        &               \\
                 &        &            & \vdots            & \ddots & \vdots              &        &                     &        &               \\
                 &        &            & a_{r_1+r_2,r_1+1} & \cdots & a_{r_1+r_2,r_1+r_2} &        &                     &        &               \\
                 &        &            &                   &        &                     & \ddots &                     &        &               \\
                 &        &            &                   &        &                     &        & a_{n-r_m+1,n-r_m+1} & \cdots & a_{n-r_m+1,n} \\
                 &        &            &                   &        &                     &        & \vdots              & \ddots & \vdots        \\
                 &        &            &                   &        &                     &        & a_{n,n-r_m+1}       & \cdots & a_{nn}
    \end{pmatrix}\]
上图中空的部分全为$0$. 现在我们首先考虑第一个分块,根据线性映射矩阵表示的定义,我们有
\begin{gather*}
    \sigma(\alpha_1)=a_{11}\alpha_1+\cdots+a_{1r_1}\alpha_{r_1}\in \spa(\alpha_1,\ldots,\alpha_{r_1}),\\
    \cdots\\
    \sigma(\alpha_{r_1})=a_{1r_1}\alpha_1+\cdots+a_{r_1r_1}\alpha_{r_1}\in \spa(\alpha_1,\ldots,\alpha_{r_1}),
\end{gather*}
有趣的事情出现了,我们发现$\sigma$作用于$\alpha_1,\ldots,\alpha_{r_1}$中每个向量都会落在$\spa(\alpha_1,\ldots,\alpha_{r_1})$中. 进一步地,我们考虑任意的$\alpha\in\spa(\alpha_1,\ldots,\alpha_{r_1})$,我们有
\[\alpha=k_1\alpha_1+\cdots+k_{r_1}\alpha_{r_1},\]
因此
\[\sigma(\alpha)=k_1\sigma(\alpha_1)+\cdots+k_{r_1}\sigma(\alpha_{r_1})\in \spa(\alpha_1,\ldots,\alpha_{r_1}),\]
即$\sigma$作用于$\spa(\alpha_1,\ldots,\alpha_{r_1})$中的任意向量仍然落在$\spa(\alpha_1,\ldots,\alpha_{r_1})$中. 同理,我们可以证明$\sigma$作用于$\spa(\alpha_{r_1+1},\ldots,\alpha_{r_1+r_2})$中的任意向量仍然落在$\spa(\alpha_{r_1+1},\ldots,\alpha_{r_1+r_2})$中,以此类推,每个分块对应的基的一部分的线性扩张都有这样类似的性质. 我们知道一组向量的线性扩张是包含这些向量的最小子空间,再结合该子空间在线性变换作用后仍然落在其中的性质,我们将其称为不变子空间:
\begin{definition}{}{}
    设$\sigma\in \mathcal{L}(V)$,若$V$的子空间$U$满足$\forall \alpha\in U,\enspace \sigma(\alpha)\in U$,则称$U$是$\sigma$的\term{不变子空间}\index{xianxingkongjian!zi!bubian@不变子空间 (invariant subspace)},或称$U$在$\sigma$下不变,简称为$\sigma$-子空间.
\end{definition}

即线性变换把不变子空间$U$中的元素全都映射到$U$自身的元素,所以``不变''这一形容词是非常形象的. 为了日后描述的方便,我们需要引入一个限制映射的概念. 我们给出标准的定义如下:
\begin{definition}{}{} \index{xianxingyingshe!xianzhi@限制 (restriction map)}
    设$V$是数域$\mathbf{F}$上的线性空间,$\sigma\in\mathcal{L}(V)$,我们在$V$的子空间$U$上定义映射$\sigma\vert_U$如下:
    \[\sigma\vert_U:U\to V,\enspace\sigma\vert_U(\alpha)=\sigma(\alpha),\enspace\forall \alpha\in U,\]
    则称$\sigma\vert_U$是$\sigma$在$U$上的\term{限制映射}.
\end{definition}

``限制''一词十分形象(如下图所示),因为限制映射就是将原映射的定义域限制在更小的范围内,但原定义域上的映射值保持不变.
\begin{figure}[H]
    \centering
    \begin{tikzpicture}[>=Stealth]
        \draw[thick] (-2,0) ellipse (1 and 1.5) node[above=20] {$V$}
        (2,0) ellipse (1 and 1.5) node[above=20] {$V$}
        (-1.9,-0.5) circle (0.5) node {$U$}
        (1.9,-0.5) ellipse (0.5 and 0.7) node {$\sigma(U)$}
        (-1.8,0.5) edge[out=30,in=150,->] node[above] {$\sigma$} (1.8,0.5)
        (-1.6,-0.6) edge[out=-30,in=190,->] node[below] {$\sigma\vert_U$} (1.75,-0.8);
    \end{tikzpicture}
\end{figure}

因此,如果$U$是$\sigma$的不变子空间,根据不变子空间的定义可知,$\sigma\vert_U$就是一个线性变换(是$\mathcal{L}(U)$中的元素),我们称之为\term{限制变换}\index{xianxingyingshe!xianxingbianhuan!xianzhi@限制变换 (restriction operator)}.

有了上面的概念,我们很容易可以将之前的推导转化为以下定理:
\begin{theorem}{}{不变子空间与分块对角矩阵}
    设有限维线性空间$V$上的线性变换$\sigma\in\mathcal{L}(V)$在某组基下的表示矩阵为分块对角矩阵$A=\diag(A_1,\ldots,A_m)$,当且仅当$V$可以分解为不变子空间$U_1,\ldots,U_m$的直和,即
    \[V=U_1\oplus\cdots\oplus U_m,\]
    其中每个$U_i$都是$\sigma$的不变子空间,且$\sigma\vert_{U_i}$在$U_i$对应的基下的表示矩阵为$A_i$.
\end{theorem}

根据定义我们可以验证或者求解一些很简单的不变子空间. 例如设$T\in \mathcal{L}(V)$,则$V$的两个平凡子空间$\{0\}$和$V$,以及映射的像与核$\ker T,\im T$都是$\sigma$不变子空间,验证非常简单,此处不赘述. 事实上当$p$为多项式时,$\ker p(\sigma)$和$\im p(\sigma)$也为$\sigma$的不变子空间. 我们这里也简要书写一下,供读者熟悉如何利用定义验证不变子空间:
\begin{example}{}{多项式不变子空间}
    若$\sigma\in\mathcal{L}(V)$且$p\in\mathbf{F}[x]$为多项式,则$\ker p(\sigma)$和$\im p(\sigma)$在$\sigma$下不变.
\end{example}

\begin{proof}
    我们只需验证$\ker p(\sigma)$和$\im p(\sigma)$中的元素经过$\sigma$映射后仍在这一空间中即可.
    \begin{enumerate}
        \item $\forall \alpha\in \ker p(\sigma),\enspace p(\sigma)\alpha=0$,因此
              \[(p(\sigma))(\sigma(\alpha))=\sigma(p(\sigma)\alpha)=\sigma(0)=0,\]
              即$\sigma(\alpha)\in \ker p(\sigma)$,因此$\ker p(\sigma)$在$\sigma$下不变;

        \item $\forall \alpha\in \im p(\sigma),\enspace \exists \beta\in V,\enspace \alpha=p(\sigma)\beta$,因此
              \[\sigma(\alpha)=\sigma(p(\sigma)\beta)=p(\sigma)(\sigma(\beta))\in \im p(\sigma),\]
              即$\im p(\sigma)$在$\sigma$下不变.
    \end{enumerate}
    事实上,对于$\ker p(\sigma)$,我们有
    \[ p(\sigma)(\alpha)=p(\sigma)\alpha=p(\sigma(\alpha))=p(0)=0,\]
    因此$\sigma(\alpha)\in \ker p(\sigma)$,即$\ker p(\sigma)$在$\sigma$下不变. 对于$\im p(\sigma)$,我们有
    \[\forall \alpha\in \im p(\sigma),\enspace \exists \beta\in V,\enspace \alpha=p(\sigma)\beta,\]
    因此
    \[\sigma(\alpha)=\sigma(p(\sigma)\beta)=p(\sigma)\sigma(\beta)\in \im p(\sigma),\]
    即$\im p(\sigma)$在$\sigma$下不变.
\end{proof}

有时我们可能会遇到更为复杂的情形,如下面的例子:
\begin{example}{}{不变子空间}
    设$\mathbf{F}$为一数域,线性变换$\sigma\in\mathcal{L}(\mathbf{F}^2)$定义为
    \[\sigma(a,b)=(a,b)\begin{pmatrix}
            1 & -1 \\ 2 & 2
        \end{pmatrix}\]
    证明:
    \begin{enumerate}
        \item 当$\mathbf{F}=\mathbf{R}$时,$\mathbf{R}^2$无$\sigma$的非零真不变子空间;

        \item 当$\mathbf{F}=\mathbf{C}$时,$\mathbf{C}^2$有$\sigma$的非零真不变子空间.
    \end{enumerate}
\end{example}

\begin{proof}
    事实上,由于$\sigma$定义在二维空间$\mathbf{F}^2$上,因此``非零真不变子空间''只能是一维子空间. 设该不变子空间$U=\spa(\alpha)(\alpha\neq 0)$,并进一步设$\alpha=(a,b)$. 我们知道,一维线性空间中所有元素都是成比例的(可以理解为一条直线,或者一维空间是由一个向量线性扩张而来,扩张过程中的线性组合一定保证后面生成的所有向量都互相成比例). 我们假设比例值为$\lambda$,即
    \[\sigma(a,b)=(a,b)\begin{pmatrix}
            1 & -1 \\ 2 & 2
        \end{pmatrix}=\lambda(a,b)=(\lambda a,\lambda b),\]
    将矩阵乘法展开,我们有
    \[(a+2b,-a+2b)=(\lambda a,\lambda b).\]
    由于$\alpha=(a,b)\neq 0$,因此$\lambda\neq 0$,基于此解方程得到$\lambda^2-3\lambda+4=0$. 这一方程在实数域范围内无解,复数域内有两个共轭的解,因此,我们有
    \begin{enumerate}
        \item 当$\mathbf{F}=\mathbf{R}$时,$\mathbf{R}^2$无$\sigma$的非零真不变子空间;

        \item 当$\mathbf{F}=\mathbf{C}$时,$\mathbf{C}^2$有$\sigma$的非零真不变子空间.
    \end{enumerate}
\end{proof}

除此之外,还有一些更为困难的问题我们将在讨论完标准形理论之后反过来进行讨论. 为了接下来对标准形讨论的方便,我们需要介绍一个特别的不变子空间:
\begin{definition}{}{循环子空间}
    设$T\in\mathcal{L}(V)$,$v\in V$是一个非零向量. 我们称子空间
    \[W=\spa(v,Tv,T^2v,\ldots,T^kv,\cdots)\]
    为\term{由$v$生成的$T\text{-循环子空间}$}. 在不引起歧义的情况下,我们也称$W$为\term{$T\text{-循环子空间}$}或\term{循环子空间}.
\end{definition}

一个需要注意的地方是,当$V$是有限维线性空间时,循环子空间也一定是有限维的(子空间的维数至少要小于等于原空间). 一个有趣的事实是,如果循环子空间的维数等于$m$,那么它的一组基就是$\{v,Tv,\ldots,T^{m-1}v\}$. 我们来书写这一定理并给出证明:
\begin{theorem}{}{循环子空间}
    设$V$是有限维线性空间,$T\in\mathcal{L}(V)$,$v\in V$是一个非零向量,$W$是由$v$生成的$T-\text{循环子空间}$,则
    \begin{enumerate}
        \item $W$是$T$包含$v$的最小不变子空间;
        \item 若$W$的维数为$m$,则$W$的一组基为$\{v,Tv,\ldots,T^{m-1}v\}$(我们称其为一组循环基).
    \end{enumerate}
\end{theorem}
\begin{proof}
    \begin{enumerate}
        \item 首先我们证明是不变子空间. 由于$V$是有限维线性空间,故$W$一定也是有限维线性空间,故任意的$w\in W$都可以被表示为$W$中有限个元素的线性组合,假设为
              \[w=k_1T^{m_1}v+k_2T^{m_2}v+\cdots+k_sT^{m_s}v,\]
              其中$m_1,m_2,\ldots,m_s\in\mathbf{N}$,$k_1,k_2,\ldots,k_s\in\mathbf{F}$. 我们有
              \[\begin{aligned}
                      Tw & =T(k_1T^{m_1}v+k_2T^{m_2}v+\cdots+k_sT^{m_s}v)     \\
                         & =k_1T^{m_1+1}v+k_2T^{m_2+1}v+\cdots+k_sT^{m_s+1}v,
                  \end{aligned}\]
              因此$Tw$仍然是$W$中向量的线性组合,即$Tw\in W$,故$W$是$T$的不变子空间.

              接下来我们证明是最小不变子空间. 设$U$是$T$的不变子空间,且$v\in U$,我们需要证明$W\subset U$. 由不变子空间的定义以及$v\in U$,我们知道$Tv\in U$,于是进一步利用不变子空间的定义有$T^2v\in U$,以此类推,$T^kv\in U,\enspace\forall k\in\mathbf{N}$,因此$W\subset U$,即$W$是$T$包含$v$的最小不变子空间.

        \item 由于$v$是非零向量,故设$j$是使得$v,Tv,\ldots,T^{j-1}v$线性无关的最大正整数,设$U=\spa(v,Tv,\ldots,T^{j-1}v)$,因此$v,Tv,\ldots,T^{j-1}v$是$U$的一组基. 并且根据我们的假设,$v,Tv,\ldots,T^{j-1}v,T^jv$线性相关,因此根据线性相关性质有
              \[T^jv=k_1v+k_2Tv+\cdots+k_{j-1}T^{j-1}v,\]
              即$T^jv\in U$,于是我们任取$u\in U$,有
              \[u=c_1v+c_2Tv+\cdots+c_{j-1}T^{j-1}v,\]
              因此
              \[Tu=c_1Tv+c_2T^2v+\cdots+c_{j-1}T^jv\in U,\]
              故$U$是$T$的包含$v$的不变子空间. 由定理第一条可知,$W\subset U$,但根据$U$的定义又有$U\subset W$,因此$W=U$,即$W$的一组基为$\{v,Tv,\ldots,T^{j-1}v\}$,故事实上$j$就是$W$的维数. 因此若$W$的维数为$m$,则$W$的一组基为$\{v,Tv,\ldots,T^{m-1}v\}$.
    \end{enumerate}
\end{proof}

最后我们再基于不变子空间讨论一个商线性变换的概念. 事实上,如果$U$是$\sigma$的不变子空间,那么$\sigma$还可以诱导出商空间$V/U$上的线性变换. 定义如下:
\begin{definition}{}{}
    设$\sigma\in \mathcal{L}(V)$,$U$是$\sigma$的不变子空间,定义映射$\sigma/U:V/U\to V/U$如下:
    \[(\sigma/U)(v+U)=\sigma(v)+U,\enspace\forall v\in V,\]
    则称$\sigma/U$是$\sigma$在$U$上的\term{商线性变换}\index{xianxingbianhuan@shang!商线性变换 (quotient operator)}.
\end{definition}

定义映射后,我们自然的想法就失确认这一定义是不是合理的. 首先这一定义的线性性容易验证,我们只需要用到商空间中定义的运算性质即可:
\begin{itemize}
    \item 齐次性:$(\sigma/U)(\lambda(v+U))=(\sigma/U)(\lambda v+U)=\sigma(\lambda v)+U=\lambda\sigma(v)+U=\lambda(\sigma/U)(v+U)$;

    \item 加性:$(\sigma/U)((v_1+U)+(v_2+U))=(\sigma/U)(v_1+v_2+U)=\sigma(v_1+v_2)+U=\sigma(v_1)+\sigma(v_2)+U=(\sigma/U)(v_1+U)+(\sigma/U)(v_2+U)$.
\end{itemize}

除了线性的要求外,还有一个很重要的合理性来源于之前在等价类和商空间中讨论的相容性(或者良定义)的概念,因为这里将线性变换定义在了等价类上. 事实上,对于一个映射,其相容性的关键在于原像集合中的同一个元素只能映射到像集中的唯一一个值(否则不符合映射的定义),具体而言,商线性变换的出发空间元素是等价类,因此如果出现$v+U=w+U$但$\sigma(v)+U\neq \sigma(w)+U$的情况,这一定义描述的就不是映射(因为映射要求一个自变量只能映到一个值上). 我们可以验证这一映射是满足相容性的:

\begin{proof}
    设$v+U=w+U$,即$v-w\in U$,由于$U$在$\sigma$下不变,则$\sigma(v-w)\in U$,即$\sigma(v)-\sigma(w)\in U$,因此$\sigma(v)+U=\sigma(w)+U$,即$\sigma/U$是满足相容性的.
\end{proof}

\section{特征值与特征向量}

在\autoref{thm:不变子空间与分块对角矩阵} 中我们得到了一个很关键的观察,就是不变子空间与分块对角矩阵的关联. 因此我们很自然地希望展开对不变子空间的研究,研究原空间是否能,以及怎么能分解为合理的不变子空间的直和,得到令人满意的简单的矩阵表示. 我们自然需要从最简单的不变子空间开始我们的研究,即一维的不变子空间.

事实上,在\autoref{ex:不变子空间} 中,我们已经尝试求解了一维不变子空间. 根据一维空间中向量都成比例的性质,设$U$是$\sigma\in\mathcal{L}(V)$的一维不变子空间,我们有
\[\exists\lambda\in\mathbf{F},\enspace\sigma(\alpha)=\lambda\alpha,\enspace\forall \alpha\in U,\]
即任意向量作用线性变换后的结果与原向量成比例. 这一性质将引入接下来的特征值与特征向量的概念,事实上它们对于获得简单矩阵的目标而言非常重要,因此是我们讨论的很好的开始.

在接下来的讨论中,我们很多定义和结论都会有相应的矩阵和映射版本. 回顾在\autoref{thm:线性映射对向量坐标的影响} 中的讨论,我们提到了矩阵$A$和线性映射$\sigma(\alpha)=A\alpha$的统一性,并且在相似的引入中我们也提到了我们的目标在线性变换和矩阵两个角度下的陈述及其关联,因此我们未来将不特别区分矩阵和线性变换,这一点在本节过后将有更深刻的体会.

\subsection{特征值与特征向量的定义与求解}

首先介绍线性变换和矩阵的特征值与特征向量的概念:
\begin{definition}{}{}
    设$\sigma$是线性空间$V(\mathbf{F})$上的一个线性变换,如果存在数$\lambda\in\mathbf{F}$和非零向量$\xi\in V$使得$\sigma(\xi)=\lambda\xi$,则称数$\lambda$为$\sigma$的一个\term{特征值}\index{tezhengzhi@特征值 (eigenvalue)},并称非零向量$\xi$为$\sigma$属于其特征值$\lambda$的\term{特征向量}\index{tezhengxiangliang@特征向量 (eigenvector)}.
\end{definition}
必须注意特征向量为非零向量,否则零向量$\xi=\vec{0}$对任意$\lambda$都满足上面定义,从而失去``特征''的含义. 但是特征值可以为0,此时$\sigma(\xi)=\vec{0}$,即全体特征向量的集合就是线性变换的核空间.

对于某一个$\lambda\in\mathbf{F}$,我们将所有满足$\sigma(\xi)=\lambda\xi$的向量构成的集合记为$E(\lambda,\sigma)=\{\xi \mid \sigma(\xi)=\lambda\xi,\enspace\xi\in V\}$(在去除线性变换不引起歧义的情况下可简写为$V_\lambda$),称为$\sigma$关于其特征值$\lambda$的\term{特征子空间}\index{tezhengzikongjian@特征子空间 (eigenspace)}. 显然,这一集合是由零向量和全体$\lambda$对应的特征向量构成的. 我们可以验证$V_\lambda$的确是$V$的``子空间'':
\begin{example}{}{}
    证明:$V_\lambda$是$V$的子空间,并且是线性变换$\sigma$的不变子空间.
\end{example}

\begin{proof}
    回顾证明子空间的两个要求:非空和运算封闭性. 首先$V_\lambda$非空,因为$\vec{0}\in V_\lambda$,即$\sigma(\vec{0})=\lambda\vec{0}$,因此$\vec{0}\in V_\lambda$,故$V_\lambda$非空.

    其次,对于任意$\xi_1,\xi_2\in V_\lambda$,$k_1,k_2\in\mathbf{F}$,我们有
    \[\sigma(k_1\xi_1+k_2\xi_2)=k_1\sigma(\xi_1)+k_2\sigma(\xi_2)=k_1\lambda\xi_1+k_2\lambda\xi_2=\lambda(k_1\xi_1+k_2\xi_2),\]
    因此$k_1\xi_1+k_2\xi_2\in V_\lambda$,故满足线性运算封闭. 综上,$V_\lambda$是$V$的子空间.

    不变子空间也是显然的. 对于任意$\xi\in V_\lambda$,因为$V_\lambda$是由零向量和全体$\lambda$对应的特征向量构成的,因此必有$\sigma(\xi)=\lambda\xi\in V_\lambda$,因此$V_\lambda$是$\sigma$的不变子空间.
\end{proof}

事实上,我们通过之后的例子会知道,$V_\lambda$的维数不一定是1,而至少是1. 那么我们之前引入特征值特征向量时所说的``一维不变子空间''是什么呢?事实上,我们可以取$V_\lambda$的任一组基$\alpha_1,\ldots,\alpha_n$,则其中任一向量进行扩张得到的子空间$U_i=\spa(\alpha_i),\enspace i=1,\ldots,n$就是一维不变子空间,因为$\forall u_i\in U_i,\enspace \sigma(u_i)=\lambda u_i\in U_i$,即$U_i$在$\sigma$下不变. 我们还需要注意,一维不变子空间的选取是不唯一的,因为$V_\lambda$的基的选取是不唯一的,因此$U_i$的选取也是不唯一的,实际上对于任意的$\alpha\in\V_{\lambda}$,$\spa(\alpha)$都是一维不变子空间.

上面是线性变换的特征值与特征向量的定义. 然而我们有一个无法绕开的问题,就是基于线性变换计算特征值与特征向量似乎并不是一个程序化的过程. 因此我们需要求助于矩阵——这是一个适合于计算的良好工具. 对应的,我们给出矩阵的特征值与特征向量的定义:
\begin{definition}{}{}
    设矩阵$A\in \mathbf{M}_n(\mathbf{F})$,如果存在数$\lambda\in\mathbf{F}$和非零向量$X\in\mathbf{F}^n$使得$AX=\lambda X$,则称数$\lambda$为$A$的一个特征值,称非零向量$X$为$A$属于其特征值$\lambda$的特征向量.
\end{definition}

下面我们说明线性映射的特征值与特征向量和矩阵的特征值与特征向量之间的关系. 实际上,假设$A$是$\sigma$在基$\alpha_1,\ldots,\alpha_n$下的表示矩阵,且$\xi=(\alpha_1,\ldots,\alpha_n)X$,即$X$是$\xi$在基$\alpha_1,\ldots,\alpha_n$下的坐标,则我们有
\begin{align*}
    \sigma(\xi)=\lambda\xi & \iff \sigma((\alpha_1,\ldots,\alpha_n)X)=\lambda(\alpha_1,\ldots,\alpha_n)X        \\
                           & \iff (\sigma(\alpha_1,\ldots,\alpha_n))X=(\lambda\alpha_1,\ldots,\lambda\alpha_n)X \\
                           & \iff (\alpha_1,\ldots,\alpha_n)AX=(\alpha_1,\ldots,\alpha_n)(\lambda X)            \\
                           & \iff AX=\lambda X
\end{align*}
其中第一行与第二行间的等价关系用到了矩阵乘法一节中证明的性质$\sigma((\alpha_1,\ldots,\alpha_n)X)=(\sigma(\alpha_1,\ldots,\alpha_n))X$. 由上述讨论可知$\lambda$同时是线性变换和矩阵的特征值,与基的选取无关. 但矩阵的特征向量$X$是线性映射特征向量在基下的坐标,这与基的选取有关. 由于基$\alpha_1,\ldots,\alpha_n$可以是任取的,于是求解线性变换的特征值的问题完全转化为了求线性变换任意一个矩阵表示的特征值,而特征向量也仅仅是向量与坐标的关系. 于是接下来我们讨论如何具体求解特征值与特征向量. 我们首先需要证明一个定理做一个简单的观察:
\begin{theorem}{}{}
    设$\sigma$是$V(\mathbf{F})$上的线性变换,$I$为恒等映射,则下述条件等价:
    \begin{enumerate}[label=(\arabic*)]
        \item \label{item:18:特征值定义:1}
              $\lambda\in\mathbf{F}$是$\sigma$的特征值;

        \item \label{item:18:特征值定义:2}
              $\sigma-\lambda I$不是单射;

        \item \label{item:18:特征值定义:3}
              $\sigma-\lambda I$不是满射;

        \item \label{item:18:特征值定义:4}
              $\sigma-\lambda I$不可逆.
    \end{enumerate}
\end{theorem}

\begin{proof}
    \begin{itemize}
        \item[\ref*{item:18:特征值定义:1}$\implies$\ref*{item:18:特征值定义:2}] $\lambda\in\mathbf{F}$是$\sigma$的特征值,说明$\exists v\in V$且$v\neq 0$使得$\sigma(v)=\lambda v$. 因此$(\sigma-\lambda I)(v)=0$,即$\sigma-\lambda I$核空间不只有零元,根据单射等价条件\autoref{thm:单射与核空间},不单成立;

        \item[\ref*{item:18:特征值定义:2}$\implies$\ref*{item:18:特征值定义:3}] 根据\autoref{thm:双射等价条件} 可知,$\sigma-\lambda I$不满当且仅当$\sigma-\lambda I$不单;

        \item[\ref*{item:18:特征值定义:3}$\implies$\ref*{item:18:特征值定义:4}] 根据\autoref*{thm:双射等价条件} 显然;

        \item[\ref*{item:18:特征值定义:4}$\implies$\ref*{item:18:特征值定义:1}] $\sigma-\lambda I$不可逆,根据\autoref*{thm:双射等价条件} 可知其不为单射,又根据单射等价条件\autoref*{thm:单射与核空间} 可知$(\sigma-\lambda I)(v)=0$有非零解,即$\sigma(v)=\lambda v$,其中$v\neq 0$,这与特征值定义一致.
    \end{itemize}
\end{proof}

由上述定理,$\lambda\in\mathbf{F}$是$\sigma$的特征值等价于$\sigma-\lambda I$不可逆,因此其在$V$的任意一组基$\alpha_1,\ldots,\alpha_n$下的矩阵$A-\lambda E$也不可逆(其中$A$为$\sigma$在这组基下的矩阵,$E$为单位矩阵),这又等价于$|A-\lambda E|=0$.

因此$\lambda\in\mathbf{F}$是$\sigma$的特征值等价于$|\lambda E-A|=0$,故我们可以通过$|\lambda E-A|=0$求解特征值,其中$A$为$\sigma$在某组基下的矩阵,$E$为单位矩阵. 对于特征向量的求解,求出$(\lambda E-A)X=0$的非零解就是特征向量在基$\alpha_1,\ldots,\alpha_n$下的坐标,如果是矩阵的特征向量,那么$X$就是解.

上述求解特征向量的方法需要我们求解$f(\lambda)=|\lambda E-A|$的根,事实上$f(\lambda)=|\lambda E-A|$是在之后的讨论中有核心地位的概念,我们称其为矩阵$A$的\term{特征多项式}{tezhengduoxiangshi@特征多项式 (characteristic polynomial)},其$k$重根称为$k$重特征值(称$k$为代数重数),该特征值对应的特征子空间维数称为该特征值的几何重数.

\begin{example}{}{}
    设$A=\begin{pmatrix}
            1 & -1 & 0 \\ 2 & 0 & 1 \\ 1 & a & 0
        \end{pmatrix}$,且存在非零向量$\alpha$使得$A\alpha=2\alpha$,求$a$.
\end{example}

\begin{solution}
    由题意知2是矩阵$A$的特征值,因此我们有
    \[|2E-A|=\begin{vmatrix}
            1 & 1 & 0 \\ -2 & 2 & -1 \\ -1 & -a & 2
        \end{vmatrix}=9-a=0,\]
    因此$a=9$.
\end{solution}

接下来,我们将特征多项式定义中的行列式展开得到以下定理:
\begin{theorem}{}{特征多项式展开}
    对于$n$级矩阵$A=(a_{ij})$,记
    \[f(\lambda)=|\lambda E-A|=a_0\lambda^n+a_1\lambda^{n-1}+\cdots+a_{n-1}\lambda+a_n\]
    则$a_0=1$,$a_1=-\tr(A)$,$a_n=(-1)^n|A|$,且$a_k$等于所有$k$级主子式之和乘以$(-1)^k$.
\end{theorem}

\begin{proof}
    设$A=(a_{ij})$的列向量为$\alpha_1,\ldots,\alpha_n$,则
    \[f(\lambda)=|\lambda E-A|=\begin{vmatrix}
            \lambda e_1-\alpha_1 & \lambda e_2-\alpha_2 & \cdots & \lambda e_n-\alpha_n
        \end{vmatrix}.\]
    其中$e_1,\ldots,e_n$为$\mathbf{F}^n$的标准基,因此根据行列式的\autoref{def:公理化定义} 的加性,$f(\lambda)$可以拆成$2^n$个行列式的和,它们是
    \begin{equation}\label{eq:18:特征多项式展开}
        (-\alpha_1,\ldots,-\alpha_{j_1-1},\lambda e_{j_1},-\alpha_{j_1+1},\ldots,-\alpha_{j_2-1},\lambda e_{j_2},-\alpha_{j_2+1},\ldots,-\lambda e_{j_{n-k}},\ldots,-\alpha_n),
    \end{equation}
    其中$1\leqslant j_1<j_2<\cdots<j_{n-k}\leqslant n,\enspace k=0,2,\ldots,n$.

    上式初看会显得非常复杂,但实际上利用行列式定义的加性去拆分就是每列有两种拆出来的选择,一种是选择$\lambda e_{j_i}$,另一种是选择$-\alpha_{j_i}$,这就是$2^n$种拆分方式的来由. 其中取出$k$列$-\alpha_{j_i}$,剩余$n-k$列选择$\lambda e_{j_i}$的就可以表示为上式的形式.

    利用\autoref{thm:Laplace定理} 对\autoref{eq:18:特征多项式展开} 第$j_1,\ldots,j_{n-k}$列展开,我们发现这$n-k$列元素组成的$n-k$阶子式只有一个不为0:
    \[\begin{vmatrix}
            \lambda & 0       & \cdots & 0       \\
            0       & \lambda & \cdots & 0       \\
            \vdots  & \vdots  & \ddots & \vdots  \\
            0       & 0       & \cdots & \lambda
        \end{vmatrix}=\lambda^{n-k},\]
    这个不等于0的$n-k$阶子式对应的代数余子式为
    \begin{align*}
         & (-1)^{(j_1+\cdots+j_{n-k})+(j_1+\cdots+j_{n-k})}(-A)
        \begin{pmatrix}
            j_1' & j_2' & \cdots & j_k' \\
            j_1' & j_2' & \cdots & j_k'
        \end{pmatrix}                             \\
         & = (-1)^kA\begin{pmatrix}
                        j_1' & j_2' & \cdots & j_k' \\
                        j_1' & j_2' & \cdots & j_k'
                    \end{pmatrix}
    \end{align*}
    其中$j_1',\ldots,j_k'$为$1,\ldots,n$中除去$j_1,\ldots,j_{n-k}$的$k$个数按递增顺序排列的结果,这一点通过余子式的定义是显然的. 因此\autoref{eq:18:特征多项式展开} 的值为
    \[(-1)^kA\begin{pmatrix}
            j_1' & j_2' & \cdots & j_k' \\
            j_1' & j_2' & \cdots & j_k'
        \end{pmatrix}\lambda^{n-k}.\]
    这实际上只是取$n-k$列$\lambda e_{j_i}$的一种情况,事实上对于所有可能的$j_1,\ldots,j_{n-k}$的取法,我们都可以得到类似的结果,因此$|\lambda E-A|$中$\lambda^{n-k}$的系数为
    \[(-1)^k\sum\limits_{1\leqslant j_1'<\cdots<j_k'\leqslant n}A\begin{pmatrix}
            j_1' & j_2' & \cdots & j_k' \\
            j_1' & j_2' & \cdots & j_k'
        \end{pmatrix}.\]
    即$a_k$等于所有$k$级主子式之和乘以$(-1)^k$,且代入$k=0,1,n$有$a_0=1$,$a_1=-\tr(A)$,$a_n=(-1)^n|A|$.
\end{proof}

这一定理的证明事实上无需掌握,这里给出证明是为了补全教材中的空缺. 这里我们主要掌握两个特例,即由韦达定理,我们有
\begin{enumerate}
    \item $\displaystyle\sum_{i=1}^{n}\lambda_i=\sum_{i=1}^{n}a_{ii}$;

    \item $\displaystyle\prod_{i=1}^{n}\lambda_i=|A|$.
\end{enumerate}
即特征值按重数求和为矩阵的迹(即矩阵对角线元素之和),特征值按重数求积为矩阵行列式. 这一结论在解决某些问题时有一定作用.

事实上,我们这里给出的特征多项式只是矩阵的特征多项式的定义,关于线性变换特征多项式的定义以及进一步讨论将在后续章节进行,我们也会说明两种特征多项式的定义是统一的.

\subsection{相似与特征值、特征向量的关联}
可能细心的读者已经发现,我们前面的讨论中的某个位置留下了一个bug. 我们在讨论线性变换与矩阵的特征值的关联时,提到我们可以取线性变换任意一组基下的表示矩阵来计算特征值,这些特征值就是线性变换的特征值. 事实上这暗示了一个很重要的性质:线性变换任意一组基下的矩阵都有相同的特征值,或者说,相似的矩阵就有相同的特征值,否则上面的讨论一定是有问题的. 我们在此叙述这一性质并给出证明:
\begin{theorem}{}{}
    相似矩阵有相同的特征多项式(逆命题不成立),即$A\sim B$有$|\lambda E-A|=|\lambda E-B|$,从而有相同的迹,行列式,特征值,但特征向量不一定相同.
\end{theorem}
\begin{proof}
    设$B=P^{-1}AP$,则$|\lambda E-B|=|\lambda E-P^{-1}AP|=|P^{-1}(\lambda E-A)P|=|P^{-1}||\lambda E-A||P|=|\lambda E-A|$. 因此$A\sim B$有$|\lambda E-A|=|\lambda E-B|$.

    我们知道特征多项式相同则特征值相同,迹等于所有特征值之和,行列式等于所有特征值之积,因此相似矩阵有相同的迹,行列式,特征值.

    相似矩阵来源于同一线性变换在不同基下的表示,因此它们的特征向量是线性变换的特征向量在不同基下的坐标,因此不一定相同.
\end{proof}

在得到这一结论后,我们同样可以定义线性变换的特征多项式:我们就可以定义其为任意矩阵表示的特征多项式.
\begin{definition}{}{}
    设$\sigma$是$V(\mathbf{F})$上的线性变换,$A$是$\sigma$在任意一组基下的矩阵,则$\sigma$的\term{特征多项式}\index{tezhengduoxiangshi@特征多项式 (characteristic polynomial)}定义为$|\lambda E-A|$.
\end{definition}

下面我们讨论一些重要的例子. 首先要引入的例子也是重要的结论,实际上在行列式一讲中已给出类似结论,但我们现在从特征值角度考虑这一结论:
\begin{example}{}{}
    回答以下两个问题:
    \begin{enumerate}
        \item \label{item:18:特征值相同:1}
              设$A,B$均为$n$阶矩阵,证明:$\lambda\neq 0$是$AB$的特征值,则$\lambda$也是$BA$的特征值;

        \item \label{item:18:特征值相同:2}
              设$A\in \mathbf{M}_{m\times n}(\mathbf{C}),\enspace B\in \mathbf{M}_{n\times m}(\mathbf{C})$,证明:
              \[ \begin{pmatrix}
                      AB & O \\ B & O
                  \end{pmatrix}\sim\begin{pmatrix}
                      O & O \\ B & BA
                  \end{pmatrix} \]
              并由此推出$AB$和$BA$非零特征值相同,且$m=n$时有$|\lambda E-AB|=|\lambda E-BA|$.
    \end{enumerate}
\end{example}

\begin{proof}
    \begin{enumerate}
        \item 设$X$是$AB$属于$\lambda$的特征向量,则$ABX=\lambda X$,因此$B(ABX)=B(\lambda X)$,即$(BA)(BX)=\lambda(BX)$,因此$BX$是$BA$属于$\lambda$的特征向量,故$\lambda$也是$BA$的特征值.

              实际上这里还有一点需要说明,就是$BX\neq 0$,否则它将不能作为特征向量. 事实上证明是简单的,假设$BX=0$,则$ABX=0$,由于$\lambda\neq 0$,因此必然有$X=0$,但这与$X$是$AB$属于$\lambda$的特征向量矛盾,因此$BX\neq 0$.

        \item 根据分块矩阵初等变换的性质,我们可以通过不断尝试选取到$P=\begin{pmatrix}
                      E_m & A \\ O & E_n
                  \end{pmatrix}$,其逆矩阵为$P^{-1}=\begin{pmatrix}
                      E_m & -A \\ O & E_n
                  \end{pmatrix}$,我们发现恰有
              \[\begin{pmatrix}
                      E_m & -A \\ O & E_n
                  \end{pmatrix}\begin{pmatrix}
                      AB & O \\ B & O
                  \end{pmatrix}\begin{pmatrix}
                      E_m & A \\ O & E_n
                  \end{pmatrix}=\begin{pmatrix}
                      O & O \\ B & BA
                  \end{pmatrix}.\]
              因此$\begin{pmatrix}
                      AB & O \\ B & O
                  \end{pmatrix}$与$\begin{pmatrix}
                      O & O \\ B & BA
                  \end{pmatrix}$相似,因此它们的特征多项式相同,即
              \[\begin{vmatrix}
                      \lambda E_m-AB & O \\ -B & \lambda E_n
                  \end{vmatrix}=\begin{vmatrix}
                      \lambda E_m & O \\ -B & \lambda E_n-BA
                  \end{vmatrix}.\]
              根据行列式的计算性质$\begin{vmatrix}
                      A & O \\ C & B
                  \end{vmatrix}=|A||B|$,我们有
              \[|\lambda E_m-AB||\lambda E_n|=|\lambda E_m||\lambda E_n-BA|,\]
              即$\lambda^n|\lambda E_m-AB|=\lambda^m|\lambda E_n-BA|$,因此$AB$和$BA$非零特征值相同,且$m=n$时有$|\lambda E-AB|=|\lambda E-BA|$.
    \end{enumerate}
\end{proof}

不难发现上述例子中 \ref*{item:18:特征值相同:2} 是 \ref*{item:18:特征值相同:1} 的推广,因为由\ref*{item:18:特征值相同:2} 我们得到了$|\lambda E-AB|=|\lambda E-BA|(\lambda\neq 0)$.

下面这个例子非常重要,在解决一些题目时使用这一结论会更便捷:
\begin{example}{}{}
    (\autoref{thm:基的选择对向量坐标的影响} 推广)设$P^{-1}AP=B$,证明:$A,B$分别属于同一特征值$\lambda$的特征向量$X$和$Y$满足$Y=P^{-1}X$.
\end{example}

\begin{proof}
    由$AX=\lambda_0 X$以及$A=PBP^{-1}$,我们有$PBP^{-1}X=\lambda_0 X$,即$BP^{-1}X=\lambda_0 P^{-1}X$,因此$P^{-1}X$是$B$属于$\lambda_0$的特征向量,即$P^{-1}X$是$B$的特征向量,即$Y=P^{-1}X$.
\end{proof}

实际上本题是本讲义\autoref{thm:基的选择对向量坐标的影响} 的推论,原因在于$P^{-1}AP=B$说明$A$和$B$是同一个线性变换(设为$\sigma$)在不同基下的矩阵,因此$X$和$Y$只是$\sigma$关于$\lambda_0$在两组基下的坐标,因此二坐标之间相差一个过渡矩阵.

最后我们谈一个拓展题型,我们考虑矩阵方程$AX-XB=O$,若$A,B$都是$n$阶方阵且$X$可逆,则方程可以改写为$X^{-1}AX=B$,即$A$与$B$相似. 事实上,这一矩阵方程的解空间的维数实际上刻画了$A$与$B$的相似程度. 我们有如下结论:
\begin{theorem}{}{}
    设$A,B$分别为数域$\mathbf{F}$上$n$阶、$m$阶方阵,$A,B$有$r$个两两不等的公共特征值,则矩阵方程$AX-XB=O$有秩为$r$的矩阵解. 反之,若数域为复数域,矩阵方程$AX-XB=O$有秩为$r$的矩阵解,则$A,B$至少有$r$个公共的特征值(计重数).
\end{theorem}

\begin{proof}
    \begin{enumerate}
        \item 设$\lambda_1,\ldots,\lambda_r$是$A$和$B$的$r$个公共特征值,$\alpha_1,\ldots,\alpha_r$为$A$相应的特征向量. 由于$\lambda_1,\ldots,\lambda_r$也是$B^\mathrm{T}$的特征值,设$\beta_1,\ldots,\beta_r$是$B^\mathrm{T}$相应的特征向量. 则$B^\mathrm{T}\beta_i=\lambda_i\beta_i$,从而$\beta_i^{\mathrm{T}}B=\lambda\beta_i^{\mathrm{T}}$. 下面我们证明$X=(\alpha_1,\ldots,\alpha_r)\cdot(\beta_1,\ldots,\beta_r)^\mathrm{T}$是$AX-XB=0$的解.

        事实上,$AX=A(\alpha_1,\ldots,\alpha_r)\cdot(\beta_1,\ldots,\beta_r)^\mathrm{T}=(\lambda_1\alpha_1,\ldots,\lambda_r\alpha_r)\cdot(\beta_1,\ldots,\beta_r)^\mathrm{T}=\lambda_1\alpha_1\beta_1+\cdots+\lambda_r\alpha_r\beta_r$,$XB=(\alpha_1,\ldots,\alpha_r)\cdot(\beta_1,\ldots,\beta_r)^\mathrm{T}B=(\alpha_1,\ldots,\alpha_r)\cdot(\lambda_1\beta_1,\ldots,\lambda_r\beta_r)=(\lambda_1\alpha_1\beta_1+\cdots+\lambda_r\alpha_r\beta_1)$. 故$AX-XB=0$.

        接下来证明$r(X)=r$.
        注意到$r(\alpha_1,\ldots,\alpha_r)=r((\beta_1,\ldots,\beta_r)^\mathrm{T})=r$,故$r(X)\leqslant r(\alpha_1,\ldots,\alpha_r)=r,r(X)\geqslant r(\alpha_1,\ldots,\alpha_r)+r((\beta_1,\ldots,\beta_r)^\mathrm{T})-r$,从而$r(X)=r$.
        \item 设$X=P\begin{pmatrix}
            E_r & 0 \\
            0 & 0
        \end{pmatrix}Q$,则$AX-XB=0$,从而$P^{-1}AP\begin{pmatrix}
            E_r & 0 \\
            0 & 0
        \end{pmatrix} =
        \begin{pmatrix}
            E_r & 0 \\
            0 & 0
        \end{pmatrix}QBQ^{-1}$.

        设$C=P^{-1}AP=\begin{pmatrix}
            C_1 & C_2 \\
            C_3 & C_4
        \end{pmatrix}$,
        $D=QBQ^{-1}=\begin{pmatrix}
            D_1 & D_2 \\
            D_3 & D_4
        \end{pmatrix}$. 则
        $\begin{pmatrix}
            C_1 & C_2 \\
            C_3 & C_4
        \end{pmatrix}\begin{pmatrix}
            E_r & 0 \\
            0 & 0
        \end{pmatrix} = \begin{pmatrix}
            E_r & 0 \\
            0 & 0
        \end{pmatrix}\begin{pmatrix}
            D_1 & D_2 \\
            D_3 & D_4
        \end{pmatrix}$,从而知$C_1=D_1$,$C_2=D_3=0$.
        由于$|\lambda E-C|=|\lambda E_r-C_1||\lambda E_{n-r}-C_4|$,$|\lambda E-D|=|\lambda E_r-D_1||\lambda E_{m-r}-D_4|$,而$|\lambda E_r-C_1|$是一个关于$\lambda$的$r$次多项式,在复数域上有$r$个根(计重数),它们是$C$的特征值,同时也为$D$的特征值. 从而$C$和$D$至少有$r$个公共的特征值. 而相似变换不改变矩阵的特征值,这表明$A,B$至少有$r$个公共的特征值.
    \end{enumerate}
\end{proof}

由此可以看出,复数域上$n$阶、$m$阶方阵$A,B$的矩阵方程$AX=XB$只有零解的充要条件是$A,B$没有公共特征值. 我们通过一个例子应用这一定理:
\begin{example}{}{}
    设$m$阶矩阵$A$与$n$阶矩阵$B$无公共复特征值,$C$为$m\times n$矩阵,则矩阵方程$AX-XB=C$存在唯一解.
\end{example}

\begin{proof}
    设$V$是所有$m\times n$矩阵构成的线性空间,定义$V$上的线性变换$\sigma(X)=AX-XB,\enspace X\in V$. 由于$A$和$B$无公共复特征值,所以$\sigma(X)=AX-XB=O$只有零解,即$\sigma$为$V$上单射,由\autoref{thm:双射等价条件} 可知$\sigma$是满射且是同构映射. 于是,对任意的$C\in V$,都存在唯一的$X_0\in V$使得$\sigma(X_0)=C$,即矩阵方程$AX-XB=C$存在唯一解$X_0$.
\end{proof}

\subsection{特征值的基本性质}

关于特征值,我们有如下基本性质:
\begin{enumerate}
    \item 设$\lambda$是线性空间$V(\mathbf{F})$上的线性变换$\sigma$的特征值,$\xi$是$\sigma$属于$\lambda$的特征向量,则
          \begin{enumerate}
              \item $k\lambda$是$k\sigma$的特征值,$\lambda^m$是$\sigma^m$的特征值,且$\xi$仍是相应特征向量;

              \item 若$f(x)=a_nx^n+a_{n-1}x^{n-1}+\cdots+a_1x+a_0$是$\mathbf{F}$上的多项式,则$f(\sigma)(\xi)=f(\lambda)\xi$;
          \end{enumerate}

    \item 设$\lambda$是$n$阶矩阵$A$的特征值,$A$可逆,则$\lambda^{-1}$是$A^{-1}$的特征值,$|A|\lambda^{-1}$是$A$的伴随矩阵$A^*$的特征值,且特征向量不变.

    \item 设$A$为$n$阶矩阵,则$A$与$A^\mathrm{T}$有相同的特征值(含重数).
\end{enumerate}

\begin{proof}
    \begin{enumerate}
        \item \begin{enumerate}
                  \item 由于$\sigma(\xi)=\lambda\xi$,则$(k\sigma)(\xi)=k\lambda\xi$,即$k\lambda$是$k\sigma$的特征值,$\xi$仍是相应特征向量.

                        而$\sigma^m(\xi)=\sigma^{m-1}(\sigma(\xi))=\sigma^{m-1}(\lambda\xi)=\lambda\sigma^{m-1}(\xi)=\cdots=\lambda^m\xi$,即$\lambda^m$是$\sigma^m$的特征值,$\xi$仍是相应特征向量.

                  \item 利用前述$\sigma^m$的相关性质,我们有
                        \begin{align*}
                            f(\sigma)(\xi) & = (a_n\sigma^n+a_{n-1}\sigma^{n-1}+\cdots+a_1\sigma+a_0I)(\xi)              \\
                                           & = a_n\sigma^n(\xi)+a_{n-1}\sigma^{n-1}(\xi)+\cdots+a_1\sigma(\xi)+a_0I(\xi) \\
                                           & = a_n\lambda^n\xi+a_{n-1}\lambda^{n-1}\xi+\cdots+a_1\lambda\xi+a_0\xi       \\
                                           & = f(\lambda)\xi.
                        \end{align*}
              \end{enumerate}

        \item 设$\xi$是$A$的特征值,即$A\xi=\lambda\xi$,则$\xi=A^{-1}A\xi=A^{-1}\lambda\xi$,即$A^{-1}\xi=\lambda^{-1}\xi$,因此$\lambda^{-1}$是$A^{-1}$的特征值,$\xi$仍是相应特征向量.

              又由于$A$可逆时$A^*=|A|A^{-1}$,根据前面关于$k\sigma$和$A^{-1}$特征值的讨论可知,$|A|\lambda^{-1}$是$A$的伴随矩阵$A^*$的特征值,$\xi$仍是相应特征向量.

        \item 我们用特征多项式证明. 实际上,$A^\mathrm{T}$的特征多项式为$|\lambda E-A^\mathrm{T}|=|(\lambda E-A)^\mathrm{T}|=|\lambda E-A|$(回忆转置不改变行列式),实际上与$A$的特征多项式完全一致,因此$A^\mathrm{T}$与$A$有相同的特征值(含重数).
    \end{enumerate}
\end{proof}

事实上,根据我们之前对线性变换特征值和矩阵特征值的讨论,我们知道上面的结论中``矩阵''和``线性变换''都可以互相替换(除了伴随矩阵没有定义相应的映射).

下面这一例子也是一些经典的结论,应当熟悉.
\begin{example}{}{特征值的性质}
    对下列矩阵$A$的特征值,能做出怎样的断言?
    \begin{enumerate}
        \item $A$可逆/$A$不可逆/$E+A$可逆/$4E+A$不可逆;

        \item $|E-A^2|=0$;

        \item $A^2=E$(对合)/$A^2=A$(幂等)/$A^k=0$(幂零);

        \item $A=\lambda_0E+B$($\lambda_0$为常数,且已知$B$的$n$个特征值为$\lambda_1,\lambda_2,\ldots,\lambda_n$);

        \item $A$为对角块矩阵,即$A=\diag(A_1,A_2,\ldots,A_m)$.
    \end{enumerate}
\end{example}

\begin{solution}
    \begin{enumerate}
        \item $A$可逆时有$|A|=\lambda_1\cdots\lambda_n\neq 0$,因此$A$的特征值都不为0. 同理,$A$不可逆同理表明存在特征值等于0,$E+A$可逆表明$-1$不是$A$的特征值,$4E+A$不可逆表明$-4$是$A$的特征值.

        \item $|E-A^2|=|E-A||E+A|=0$,因此$\pm 1$都是$A$的特征值.

        \item 我们首先考虑对合矩阵,接下来的同理可以得到类似结论. 由于$A^2=E$,设$AX=\lambda X$,则$A^2X=\lambda^2X=X$,因此$\lambda^2=1$,即$\lambda=\pm 1$,因此$1$或$-1$是$A$的特征值.

              但这里我们需要强调的是,不同于前两问,前两问中我们都是说某些值是$A$的特征值,但无法保证$A$的特征值只能是某些值,但在本题这样给出矩阵方程的情况下,我们可以得到$A$的特征值只能是$\pm 1$,没有其他值. 我们用反证法,假设存在$\lambda_0\neq\pm 1$是$A$的特征值,即$AX=\lambda_0X$,则$A^2X=\lambda_0^2X\neq X$(因为$X$不是零向量),导出矛盾. 当然有同学可能会思考,$A$的特征值一定兼有$\pm 1$吗,事实上并非如此,例如$E$满足$E^2=E$,但其特征值只有1,$-E$满足$(-E)^2=E$,但其特征值只有$-1$. 并且利用下一讲对角化的结论可以知道(我们放在下一讲习题中供读者练习),满足$A^2=E$且特征值只有1的矩阵只能是$E$,特征值只有$-1$的矩阵只能是$-E$.

              注:本题解决过程中告诉我们一个解题技巧,如果看到$A$的多项式$f(A)=O$这种形式的表达式,事实上$A$的特征值只能是$f(\lambda)=0$的根,如上题中$f(A)=A^2-E$,则$f(\lambda)=\lambda^2-1$,因此$A$的特征值只能是$\pm 1$.

              同理,我们可以知道幂等矩阵的特征值只能是0和1,幂零矩阵的特征值只能是0(这是一个重要的幂零矩阵等价条件,未来我们会再次遇到).

        \item 设$BX=\lambda_iX_i(X_i\neq 0,\enspace i=1,\ldots,n)$,则$AX_i=\lambda_0X_i+BX_i=\lambda_0X_i+\lambda_iX_i=(\lambda_0+\lambda_i)X_i$,因此$\lambda_0+\lambda_i\enspace(i=1,\ldots,n)$都是$A$的特征值.

        \item \begin{align*}
                  |\lambda E-A| & =\begin{vmatrix}
                                       \lambda E_1-A_1 & 0               & \cdots & 0               \\
                                       0               & \lambda E_2-A_2 & \cdots & 0               \\
                                       \vdots          & \vdots          & \ddots & \vdots          \\
                                       0               & 0               & \cdots & \lambda E_m-A_m
                                   \end{vmatrix}
                                & =\prod_{i=1}^{m}|\lambda E_i-A_i|=0
              \end{align*}
              因此,$A_i,\enspace i=1,\ldots,m$的特征值都是$A$的特征值.
    \end{enumerate}
\end{solution}

当然在本题中,我们看到了一些特殊的矩阵,如对合矩阵,幂等矩阵,幂零矩阵等,我们在之后还会讨论它们的一些性质,特别是幂等矩阵和幂零矩阵,因此我们很有必要为这些矩阵写下一个定义. 当然这些名词还有对应的线性变换版本,下面我们给出正式的定义:
\begin{definition}{}{}
    一个矩阵$A$(或线性变换$\sigma$)称为\textbf{对合矩阵}(或\textbf{对合变换}),如果$A^2=E$(或$\sigma^2=I$);称为\textbf{幂等矩阵}(或\textbf{幂等变换}),如果$A^2=A$(或$\sigma^2=\sigma$);称为\textbf{幂零矩阵}(或\textbf{幂零变换}),如果存在自然数$k$使得$A^k=O$(或$\sigma^k=O$).
\end{definition}

基于上面给出的性质和例子,我们可以进一步运用特征值的性质来求解一些问题,下面是一些例子:
\begin{example}{}{}
    回答以下问题:
    \begin{enumerate}
        \item 设$\alpha=(1,0,-1)^\mathrm{T}$,且$A=\alpha\alpha^\mathrm{T}$,求$|6E-A^n|$;

        \item 设$A$为三阶矩阵,其特征值为$1,-2,-1$,求$|A|$,$A^*+3E$的特征值,$(A^{-1})^2+2E$的特征值以及$|A^2-A+E|$;

        \item 设$A$为三阶矩阵,$A^2-A-2E=O$,$|A|=2$,求$|A^*+3E|$;

        \item 设$A$为三阶矩阵,其特征值为$-1,-1,5$,求$A_{11}+A_{22}+A_{33}$;
    \end{enumerate}
\end{example}

\begin{solution}
    \begin{enumerate}
        \item 事实上$A=\alpha\alpha^\mathrm{T}=\begin{pmatrix}
                      1 & 0 & -1 \\ 0 & 0 & 0 \\ -1 & 0 & 1
                  \end{pmatrix}$,由$|\lambda E-A|=0$解得$A$的特征值为$\lambda_1=\lambda_2=0,\lambda_3=2$,而根据$A^n$的特征值性质和\autoref{ex:特征值的性质} 可知,$6E-A^n$的特征值为$6-\lambda_1^n,6-\lambda_2^n,6-\lambda_3^n$,即$6,6,6-2^n$,因此$|6E-A^n|=6^2(6-2^n)=36(6-2^n)$.

        \item 由于$A$的特征值为$1,-2,-1$,因此$|A|=1\times(-2)\times(-1)=2$,而$A^*$的特征值为$|A|\lambda^{-1}$,因此$A^*$的特征值为$2,-1,-2$,故$A^*+3E$的特征值为$A^*$的特征值加3(根据\autoref{ex:特征值的性质}),即为$5,2,1$,又根据$A^{-1}$和$A^2$特征值的性质可知,$(A^{-1})^2+2E$的特征值为$1^2+2,(-1/2)^2+2,(-1)^2+2$,即为$3,9/4,3$,而$A^2-A+E$的特征值根据$f(\sigma)$特征值性质的讨论可知为$1^2-1+1,(-2)^2-(-2)+1,(-1)^2-(-1)+1$,即为$1,7,3$,因此$|A^2-A+E|=1\times 7\times 3=21$.

        \item 设$AX=\lambda X(X\neq 0)$,则$(A^2-A-2E)X=(\lambda^2-\lambda-2)X=O$,因此$\lambda=-1$或$\lambda=2$,根据\autoref{ex:特征值的性质} 中关于对合矩阵的讨论可知,$A$的特征值恰为-1和2. 又$|A|=2$,且$A$为3阶矩阵,因此$A$的3个特征值必为-1,-1,2.

              又$A^*$的特征值为$|A|\lambda^{-1}$,因此$A^*$的特征值为$1,-2,-2$,又根据\autoref{ex:特征值的性质} 的结论,$A^*+3E$的特征值为$A^*$的特征值加3,即$\lambda_1=\lambda_2=1,\lambda_3=4$,故$|A^*+3E|=\lambda_1\lambda_2\lambda_3=4$.

        \item 由题意知$|A|=5$,故$A^*$的特征值为$|A|\lambda^{-1}$即为$\mu_1=\mu_2=-5,\mu_3=1$,而$A_{11}+A_{22}+A_{33}$就是$A^*$的迹(即矩阵对角线元素之和),因此$A_{11}+A_{22}+A_{33}=\mu_1+\mu_2+\mu_3=-9$.
    \end{enumerate}
\end{solution}

\subsection{特征向量的基本性质}

这一部分的定理与下一讲中得到简单矩阵的可对角化的等价条件直接相关,实际上有了本节的定理,可对角化条件是很显然的.
\begin{theorem}{}{特征向量的基本性质}
    设$V$是有限维的,$\sigma\in L(V)$且$\lambda\in\mathbf{F}$,则
    \begin{enumerate}
        \item $\sigma$的不同特征值对应的特征向量线性无关;

        \item $\sigma$的不同特征值对应的特征子空间的和为直和;

        \item $\sigma$最多有$\dim V$个不同的特征值.
    \end{enumerate}
\end{theorem}

\begin{proof}
    \begin{enumerate}
        \item 设$\lambda_1,\ldots,\lambda_m$是$\sigma$的互异特征值,$\xi_1,\ldots,\xi_m$是相应的特征向量. 反证法,我们假设$\xi_1,\ldots,\xi_m$线性相关,由\autoref{lem:线性相关性引理} 可知,存在$k$是使得
              \[\xi_k\in\spa(\xi_1,\ldots,\xi_{k-1})\]
              成立的最小整数,则存在$c_1,\ldots,c_{k-1}$使得
              \begin{equation}\label{eq:18:特征向量线性无关}
                  \xi_k=c_1\xi_1+\cdots+c_{k-1}\xi_{k-1}.
              \end{equation}
              将$\sigma$作用到上式两边,我们有
              \[\lambda_k\xi_k=c_1\lambda_1\xi_1+\cdots+c_{k-1}\lambda_{k-1}\xi_{k-1}.\]
              将\autoref{eq:18:特征向量线性无关} 两边乘以$\lambda_k$,然后减去上式,我们有
              \[0=c_1(\lambda_k-\lambda_1)\xi_1+\cdots+c_{k-1}(\lambda_k-\lambda_{k-1})\xi_{k-1}.\]
              由于我们选取的$k$是满足$\xi_k\in\spa(\xi_1,\ldots,\xi_{k-1})$的最小整数,因此$\xi_1,\ldots,\xi_{k-1}$线性无关,故$a_1=\cdots=a_{k-1}=0$,因此$\xi_k=0$,这与$\xi_k$是特征向量矛盾,因此$\xi_1,\ldots,\xi_m$线性无关.

        \item 回忆直和的证明方法,我们在\autoref{thm:直和等价命题} 中选取合适等价命题进行证明. 假设
              \begin{equation}\label{eq:18:特征子空间直和}
                  \xi_1+\cdots+\xi_m=0,
              \end{equation}
              其中$\xi_i\in V_{\lambda_i}$,由于$\sigma$的不同特征值对应的特征向量线性无关,因此$\xi_1,\ldots,\xi_m$不可能是特征向量,否则由\autoref{eq:18:特征子空间直和} 可知它们线性相关,故必有$\xi_1=\cdots=\xi_m=0$,这表明$\sigma$的不同特征值对应的特征子空间的和为直和.

        \item 设$\lambda_1,\ldots,\lambda_m$是$\sigma$的互异特征值,$\xi_1,\ldots,\xi_m$是相应的特征向量. 前面已经证明了$\xi_1,\ldots,\xi_m$线性无关,因此$\dim V\geqslant m$,得证.
    \end{enumerate}
\end{proof}

上述定理有如下推论:
\begin{enumerate}
    \item 若$\lambda_1,\ldots,\lambda_m$是线性映射$\sigma$互异的特征值,则$V_{\lambda_i}\cap\sum\limits_{j\neq i}V_{\lambda_j}=\{0\}
              \enspace(i=1,\ldots,m)$,则一个特征向量不能属于多个特征值. 这一推论来源于直和的一个等价条件,线性空间运算一讲的习题中有涉及.

    \item $\sigma$的不同特征值$\lambda_1,\ldots,\lambda_m$对应的特征子空间$V_{\lambda_1},\ldots,V_{\lambda_m}$的基向量合在一起构成的向量组线性无关,且是$V_{\lambda_1}+V_{\lambda_2}+\cdots+V_{\lambda_m}$的基.
\end{enumerate}

接下来这个定理讨论了代数重数和几何重数间的关系:
\begin{theorem}{}{代数重数与几何重数}
    $n$维线性空间$V(\mathbf{F})$的线性变换$\sigma$的每个特征值$\lambda_0$的重数(代数重数)大于等于其特征子空间$V_{\lambda_0}$的维数(几何重数).
\end{theorem}

\begin{proof}
    根据线性变换和矩阵特征值的统一性(即特征多项式一致,故特征值代数重数一致)以及特征向量通过坐标映射一一对应的性质(即几何重数一致),我们只需要讨论$\sigma$在$V$的某一组基下的表示矩阵$A$的情况即可.

    设$\lambda_0$对应的特征子空间维数为$r$,则存在$V_{\lambda_0}$的一组基$\xi_1,\ldots,\xi_r$,并将其扩充为$V$的一组基$\xi_1,\ldots,\xi_r,\xi_{r+1},\ldots,\xi_n$.

    定义$n$阶可逆矩阵$U=(\xi_1,\ldots,\xi_r,\xi_{r+1},\ldots,\xi_n)$,根据$A\xi_i=\lambda_0\xi_i\enspace(i=1,\ldots,r)$,我们有
    \begin{align*}
        A(\xi_1,\ldots,\xi_r,\xi_{r+1},\ldots,\xi_n) & = (\lambda_0\xi_1,\ldots,\lambda_0\xi_r,A\xi_{r+1},\ldots,A\xi_n) \\
                                                     & = (\xi_1,\ldots,\xi_r,\xi_{r+1},\ldots,\xi_n)
        \begin{pmatrix}
            \lambda_0 E_r & B \\ O & C
        \end{pmatrix}
    \end{align*}
    其中$B$是$r\times(n-r)$矩阵,$C$是$(n-r)\times(n-r)$矩阵,$O$是零矩阵. 记$D=\begin{pmatrix}
            \lambda_0 E_r & B \\ O & C
        \end{pmatrix}$,则$AU=UD\implies A=UDU^{-1}$.

    考虑特征多项式$|\lambda E-A|=|\lambda E-UDU^{-1}|=|U(\lambda E_n-D)U^{-1}|=|U||\lambda E_n-D||U^{-1}|=|\lambda E_n-D|$,故$|\lambda E-A|=|\lambda E_n-D|$. 进一步地,$|\lambda E-D|=|\lambda E_r-\lambda_0 E_r||\lambda E_{n-r}-C|=(\lambda-\lambda_0)^r|\lambda E_{n-r}-C|$,因此$\lambda_0$作为特征多项式$|\lambda E-A|$的根的重数至少为$r$,即$\lambda_0$的代数重数大于等于其特征子空间$V_{\lambda}$的维数.
\end{proof}

事实上,由于$n$阶矩阵的特征多项式是$n$次的,因此所有特征值的代数重数之和等于$n$,但是根据上述定理可知所有特征值的几何重数之和小于等于$n$,即所有特征子空间的直和不一定能够得到原空间$V$. 这将构成我们接下来讨论的一个核心:我们在下一讲中将要讨论代数重数和几何重数相等情况下的最简单的矩阵表示,以及二者不相等的时候如何对原空间进行分解(因为此时$V$不能被分解为特征子空间直和)使得我们可以获得较为简单的矩阵表示.

最后我们再通过一个例子体会特征向量和特征子空间的一些性质:
\begin{example}{}{}
    设$V(\mathbf{F})$是$n$维线性空间,$\sigma\in \mathcal{L}(V)$,证明:
    \begin{enumerate}
        \item 若$\alpha,\beta$是$\sigma$的属于不同特征值的特征向量,则$c_1c_2\neq 0$时,$c_1\alpha+c_2\beta$不是$\sigma$的特征向量;

        \item $V$中的每一非零向量都是$\sigma$的特征向量$\iff\sigma=c_0I_V$,其中$c_0\in\mathbf{F}$是一个常数,$I_V$是恒等变换.
    \end{enumerate}
\end{example}

\begin{proof}
    \begin{enumerate}
        \item 设$\sigma(\alpha)=\lambda_1\alpha,\sigma(\beta)=\lambda_2\beta$,其中$\lambda_1\neq\lambda_2$,并假设$c_1\alpha+c_2\beta$是$\sigma$的特征向量,即存在$\lambda_0\in\mathbf{F}$使得
              \[\sigma(c_1\alpha+c_2\beta)=\lambda_0(c_1\alpha+c_2\beta).\]
              展开括号,我们有
              \[c_1\sigma(\alpha)+c_2\sigma(\beta)=c_1\lambda_0\alpha+c_2\lambda_0\beta.\]
              即$c_1\lambda_1\alpha+c_2\lambda_2\beta=c_1\lambda_0\alpha+c_2\lambda_0\beta$,即$(\lambda_1-\lambda_0)c_1\alpha+(\lambda_2-\lambda_0)c_2\beta=0$,由于$\alpha,\beta$线性无关,因此
              \[c_1(\lambda_1-\lambda_0)=c_2(\lambda_2-\lambda_0)=0.\]
              当$c_1c_2\neq 0$时,我们有$\lambda_1=\lambda_0=\lambda_2$,这与$\lambda_1\neq\lambda_2$矛盾,因此$c_1\alpha+c_2\beta$不是$\sigma$的特征向量.

        \item 右推左显然,我们只考虑左推右的证明. 由上一小问结论可知,若$V$中的每一非零向量都是$\sigma$的特征向量,$\sigma$不可能有不同的特征值(因为有不同的特征值就有不同特征值对应的特征向量,但它们的线性组合一定仍在$V$中,这与从第一问中得到的结论,即它不是$\sigma$的特征向量矛盾). 设$c_0$是$\sigma$的唯一的特征值,则对于任意$\alpha\in V$,我们有$\sigma(\alpha)=c_0\alpha$,即$\sigma$在任意元素上的像都已经唯一确定,则显然在$V$的一组基上的像也唯一确定,由\autoref{thm:线性映射唯一确定} 可知这样的线性映射是唯一的,$\sigma=c_0I_V$符合要求,因此它就是我们要找的线性映射.
    \end{enumerate}
\end{proof}

事实上,本题的结论是十分具有启发性的. 它表明,即便所有特征子空间的直和等于全空间$V$,这也不表明$V$中所有向量都是特征向量,只有特征值唯一时才能做到这一点. 原因在于不同特征子空间之间是直和,因此我们无法通过两个特征子空间的基向量的线性组合(系数非零)来得到任意特征子空间中的向量,相反,这样的线性组合会使得得到的新向量不在任何一个特征子空间中,因此无法使得$V$中所有向量都是特征向量.

下面我们通过一个例子给出一种由特征向量出发生成线性无关向量组的方法,这一例子将在后续的讨论中起到重要作用:
\begin{example}{}{特征向量生成线性无关组}
    设 $A$ 是数域 $\mathbf{F}$ 上一个 $n$ 阶方阵,$E$ 是 $n$ 阶单位矩阵,$\alpha_1 \in \mathbf{F}^n$ 是 $A$ 的属于特征值 $\lambda$ 的一个特征向量,向量组 $\alpha_1,\alpha_2,\ldots,\alpha_s$ 按如下方式产生:$(A-\lambda E)\alpha_{i+1}=\alpha_i,\enspace i=1,2,\ldots,s-1$. 证明向量组 $\{\alpha_1,\alpha_2,\ldots,\alpha_s\}$ 线性无关.
\end{example}

\begin{proof}
    由于$\alpha_1$是$A$属于特征值$\lambda$的特征向量,故有$(A-\lambda E)\alpha_1=0$.

    设$\displaystyle\sum_{i=1}^{s}k_i\alpha_i=0$,两边同时左乘$A-\lambda E$可知$(A-\lambda E)\displaystyle\sum_{i=1}^{s}k_i\alpha_i=\displaystyle\sum_{i=1}^{s}k_i(A-\lambda E)\alpha_i=k_1(A-\lambda E)\alpha_1+\displaystyle\sum_{i=1}^{s-1}k_{i+1}\alpha_i=\displaystyle\sum_{i=1}^{s-1}k_{i+1}\alpha_i=0$.

    以此类推,在等式两边不断左乘$(A-\lambda E)$可知:对于$\forall r \in \{1,\cdots,s-1\}$都有$\displaystyle\sum_{i=1}^{s-r}k_{i+r}\alpha_i=0$.

    令$r=s-1$得到$k_s\alpha_1=0,k_s=0$. 再依次代回不难得到$k_i=0,\forall i \in \{1,\cdots,s\}$,从而向量组$\alpha_1,\cdots,\alpha_s$线性无关.
\end{proof}

\section{实数域与复数域的讨论}

在上一节中我们并没有明确区分特征值所在的数域(即线性空间$V$定义的数域). 实际上上面的讨论都是与数域无关的,即无论是什么数域上面的定理都是成立的. 然而,从\autoref{ex:不变子空间} 中我们看到实数域和复数域可能有本质的不同,即特征值的存在性可能存在差别. 事实上,这是\nameref{thm:多项式的唯一分解定理}的必然结果,因为复数域上$n$次多项式一定有$n$个根,但实数域上可能根会减少,因此$n$次特征多项式$f(\lambda)$在实数域上解的情况与复数域有差别.

因此我们有必要分别讨论在复数域和实数域条件下特征值与特征向量的不同性质,事实上我们将在实空间上的线性变换一讲中单独深入讨论这一主题,但现在我们需要几个定理来引入这一话题并为接下来的讨论作准备:
\begin{theorem}{}{复数域上的特征值}
    设$\sigma\in \mathcal{L}(V)$,$V$是$n$维复线性空间,则$\sigma$必有特征值.
\end{theorem}

这一定理从解特征多项式求特征值的角度来看是非常显然的,因为此时特征多项式$f(\lambda)$展开后为$n$次多项式,则由代数学基本定理,$f(\lambda)=0$在复数域上有$n$个解,因此复线性空间上的线性变换一定有特征值. 注意实线性空间上不一定有特征值,因为$f(\lambda)=0$可能无实根.

这一命题也可以不使用特征多项式解决,下面我们给出一种不使用行列式、特征多项式的证明方法:
\begin{proof}
    对于$v \in V,v \neq 0$,$v,\sigma(v),\cdots,\sigma^n(v)$线性相关,从而存在不全为$0$的复数$a_0,a_1,\cdots,a_n$,使得$a_0v+a_1\sigma(v)+\cdots+a_n\sigma^n(v)=0$. 由于$v \neq 0$,故$a_1,\cdots,a_n$不全为$0$.

    令$f(z)=\displaystyle\sum_{i=0}^{n}a_iz^i=c\displaystyle\prod_{i=1}^{m}(z-\lambda_i)$,则$0=f(\sigma)(v)=c(\sigma-\lambda_1I)\cdots(\sigma-\lambda_mI)v$,这表明$\exists j\in \{1,\cdots,m\}$使得$(\sigma-\lambda_jI)$不是单的,从而$\sigma$有本征值.
\end{proof}
当然还有很多不同的证明方法,此处篇幅有限不再赘述.

\begin{theorem}{}{特征值与不变子空间}
    任取$\sigma\in \mathcal{L}(V)$,$V$是$n$维线性空间(无论数域是实或复),则$\sigma$一定有一维或二维不变子空间.
\end{theorem}

\begin{proof}
    由\autoref{thm:复数域上的特征值} 可知,复空间$\sigma$有特征值$\lambda$,因此根据在特征子空间的讨论可知必然存在一维不变子空间.

    若$\sigma$定义在实空间上,我们可以首先考虑复数域上的特征值,若$a+b\i$是$\sigma$的特征值,其中$a,b\in\mathbf{R}$,则存在不全为零的实向量$\alpha,\beta$使得$\alpha+\beta\i$是$\sigma$的特征向量,即我们有
    \[\sigma(\alpha+\beta\i)=(a+b\i)(\alpha+\beta\i).\]
    展开括号,我们可以得到
    \[\sigma(\alpha)=a\alpha-b\beta,\sigma(\beta)=b\alpha+a\beta.\]
    令$U=\spa(\alpha,\beta)$,则$U$是$\sigma$的不变子空间,且$\dim U=1$或$\dim U=2$,具体取值取决于$\alpha$和$\beta$是否线性相关.
\end{proof}

这里讨论实空间的情况时,我们用到了一个很特别的思想,即首先考虑了复特征值和特征向量,然后通过将复数表示为$a+b\i(a,b\in\mathbf{R})$的形式转回了实空间上的研究. 这一思想我们称之为``复化'',我们将在本讲义后面的章节中更为完整地讨论这一思想.

最后我们讨论实数特征值和复数特征值几何意义的不同. 比较显然的一点是,实数域上的特征值与特征向量的几何意义在于,某一线性变换的特征向量在经过变换后得到的向量与原先向量共线,因为若$\alpha\in V$为$\sigma$的特征向量,则存在$\lambda\in\mathbf{R}$有$\sigma(\alpha)=\lambda\alpha$,因此$\alpha$被线性变换作用后相当于简单的按比例伸缩.

但是如果特征值是复数,那么情况并不会这么简单. 我们接下来的讨论思路比较直观,不够严谨,但是可以帮助我们理解复数特征值的几何意义. 我们首先来看一个例子:
\begin{example}{}{}
    设$\sigma\in\mathcal{L}(\mathbf{F}^2)$定义为$\sigma(w,z)=(-z,w)$.
    \begin{enumerate}
        \item 当$\mathbf{F}=\mathbf{R}$时,求$\sigma$的特征值和特征向量;

        \item 当$\mathbf{F}=\mathbf{C}$时,求$\sigma$的特征值和特征向量.
    \end{enumerate}
\end{example}

\begin{solution}
    我们首先写出$\sigma$在任意一组基下的矩阵表示,为了方便,我们选取标准基$e_1=(1,0),e_2=(0,1)$,则矩阵表示为
    \[ A=\begin{pmatrix}
            0 & -1 \\ 1 & 0
        \end{pmatrix}. \]
    则其特征多项式$f(\lambda)=|\lambda E-A|=\lambda^2+1$,因此
    \begin{enumerate}
        \item 在实数域上无特征值和特征向量;

        \item 复数域上特征值为$\pm\i$,其中$\i$对应的特征向量为$(w,-wi)$,$-\i$对应的特征向量为$(w,wi)$,其中$w\in\mathbf{R}$且$w\neq 0$.
    \end{enumerate}
\end{solution}

这里需要强调的一点是,$\mathbf{C}^2$也是二维线性空间,原因在于这里的$\mathbf{C}^2$的含义是定义在复数域上的,即是$\mathbf{C}^2(\mathbf{C})$,而不是$\mathbf{C}^2(\mathbf{R})$,因此维数为2而非4. 事实上在\autoref{ex:不同数域的维数} 中读者应当就已经理解了这一点,此处不再做详细解释.

事实上,我们可以抛开程序式的解题步骤,仔细观察这里的映射定义,我们会发现,在实数域内这一变换$\sigma$就是二维平面中将向量绕原点逆时针旋转90$^\circ$的旋转变换,因此在实数域内无特征值(实特征值实际上只能将特征向量沿着原方向伸缩). 但为何复数域内有特征值呢?我们回忆复数的极坐标表示,任意复数$z$可表示为$z=re^{\i\theta}$,因此直观而言复特征值除了伸缩效应外也有旋转的效应.

本题中两个特征向量可以写为$\alpha\pm \i\beta$,则$T$在$(\alpha,\beta)$这组基下的矩阵表示就是一个表示旋转90$^\circ$的矩阵乘以单位矩阵(表明伸缩为比例1),这表明线性变换对空间的伸缩作用与特征值模长对应,旋转作用与辐角对应(本题特征值$\pm \i=1\cdot(\cos 90^\circ\pm \i\sin 90^\circ)$).

我们还可以延伸到三维空间. 设三阶矩阵$A=(a_{ij})_{3\times 3}$,设这一矩阵有三个互异特征值,则根据多项式的性质可知,其中两个为共轭复数$\lambda_{1,2}=a\pm b$,还有一个实数$\lambda_3=c$,对应的特征向量为$v_{1,2}=\alpha\pm \i\beta,v_3=\gamma$,则$T$在$\alpha,\beta,\gamma$下的矩阵表示为
\[ B=\begin{pmatrix}
        a & b & 0 \\ -b & a & 0 \\ 0 & 0 & c
    \end{pmatrix}, \]
我们令$r=\sqrt{a^2+b^2},a=r\cos\theta,b=r\sin\theta$,则有
\[ B=\begin{pmatrix}
        \cos\theta & -\sin\theta & 0 \\ \sin\theta & \cos\theta & 0 \\ 0 & 0 & 1
    \end{pmatrix}\begin{pmatrix}
        r & 0 & 0 \\ 0 & r & 0 \\ 0 & 0 & c
    \end{pmatrix}. \]
我们可以看到,这个变换被分解为两个变换,一个是在$x-y$平面上的旋转,另一个是拉伸,在$x-y$平面上拉伸$r$倍,$z$方向拉伸$c$倍. 这显然是二维结论的自然推广.

在更高维的情况也是类似的,矩阵也可以表示为一个旋转向量的矩阵乘以一个伸缩向量的矩阵,旋转角度是复特征值的辐角,伸缩倍数是复特征值的模长.

\section{特征值的估计}

对于低阶的矩阵来说,我们可以通过解特征多项式来精确求得特征值;但对于高阶矩阵而言,解特征多项式是非常困难的,所幸相关的工作一般来说也不需要我们精确求得特征值,所以我们可以通过一些方法来估计特征值.

\begin{definition}{Gershgorin 圆盘}{Gershgorin disks}
    设 $T \in \mathcal{L}(V)$,并且 $v_1, \ldots, v_n$ 是 $V$ 的一组基,$A = (a_{ij})_{n \times n}$ 为 $T$ 在这组基下的矩阵表示. 那么 $T$ 关于这组基的一个 Gershgorin 圆盘是指如下形式的集合:
    \[ D_i = \left\{z \in \mathbf{F} \mid \lvert z - a_{ii} \rvert \leqslant \sum_{j \neq i} \lvert a_{ij} \rvert\right\}, i = 1, \ldots, n. \]
\end{definition}

因为有 $n$ 个对角元可供选择,所以 $T$ 共有 $n$ 个 Gershgorin 圆盘. 在复数域考虑的话,上述的 $D_i$ 就是闭圆盘,而在实数域考虑的话,$D_i$ 就是闭区间.

在进一步讨论之前,先考虑一种特殊的情况:对角矩阵. 设 $T \in \mathcal{L}(V)$,并且 $v_1, \ldots, v_n$ 是 $V$ 的一组基,$A = (a_{ij})_{n \times n}$ 为 $T$ 在这组基下的矩阵表示. 若 $A$ 是对角矩阵,那么 $T$ 相应的 Gershgorin 圆盘就是 $n$ 个单点,也就是对应的特征值. 这让我们朦胧之中觉得 Gershgorin 圆盘或许可以“控制”特征值. 事实上,这一想法是正确的:

\begin{theorem}{Gershgorin 圆盘第一定理}{Gershgorin disks theorem}
    设 $T \in \mathcal{L}(V)$,并且 $v_1, \ldots, v_n$ 是 $V$ 的一组基. 那么 $T$ 的每个特征值都在其关于 $v_1, \ldots, v_n$ 这组基的某个 Gershgorin 圆盘内.
\end{theorem}

\begin{proof}
    设 $\lambda \in \mathbf{F}$ 是 $T$ 的一个特征值,$w \in V$ 是对应的一个特征向量,所以存在 $c_1,\ldots, c_n \in \mathbf{F}$ 使得
    \[
        w = c_1 v_1 + \cdots + c_n v_n.
    \]
    设 $A$ 是 $T$ 关于 $v_1, \ldots, v_n$ 这组基的矩阵表示,对上式两侧同时作用 $T$,便有
    \begin{align*}
        \lambda w & = Tw = \sum_{i = 1}^n c_i T v_i                                \\
                  & = \sum_{i = 1}^n c_i \sum_{j = 1}^n a_{ji} v_j                 \\
                  & = \sum_{j = 1}^n \left( \sum_{i = 1}^n a_{ji} c_i \right) v_j.
    \end{align*}
    设 $j$ 是使得 $\lvert c_j \rvert = \max_{1 \leqslant i \leqslant n} \lvert c_i \rvert$ 的下标,那么结合 $w$ 在这组基下的展开式,我们有
    \[
        \lambda c_j = \sum_{i = 1}^n a_{ji} c_i.
    \]
    进而在两边减去 $a_{jj} c_j$,并且除以 $c_j$,可以得到
    \begin{align*}
        \lvert \lambda - a_{jj} \rvert & = \left\lvert \sum_{i \neq j} a_{ji} \frac{c_i}{c_j} \right\rvert                          \\
                                       & \leqslant \sum_{i \neq j} \lvert a_{ji} \rvert \frac{\lvert c_i \rvert}{\lvert c_j \rvert} \\
                                       & \leqslant \sum_{i \neq j} \lvert a_{ji} \rvert.
    \end{align*}
    所以 $\lambda$ 位于 $T$ 的关于 $v_1, \ldots, v_n$ 这组基的第 $j$ 个 Gershgorin 圆盘内.
\end{proof}

依据这条定理,我们成功地将特征值的可行区域从复平面或实数轴缩小到了有限个 Gershgorin 圆盘. 而根据对角矩阵的例子,很自然便会考虑特征值与 Gershgorin 圆盘之间是否存在一一对应的关系,即一个 Gershgorin 圆盘内必有一个特征值(重数按照特征值的代数重数计算). 遗憾的是,这一定理是不成立的.

\begin{example}{}{}
    设 $A = \begin{pmatrix}
            1           & -\dfrac{4}{5} & 0  \\
            \dfrac{1}{2} & 0            & 0  \\
            0           & 0            & \i
        \end{pmatrix}$,求出其特征值与 Gershgorin 圆盘,并在复平面上进行标注.
\end{example}

不过我们可以看看这个例子的特殊之处,它的其中两个 Gershgorin 圆盘是相交的,所以尽管对应的两个特征值都落在了同一个圆盘内,但也可以被描述为其特征值落在了两个圆盘构成的连通区域内. 而考虑到对角矩阵对应的 Gershgorin 圆盘是 $n$ 个单点,除了重合外没有连通的情况,会出现一个 Gershgorin 圆盘对应一个特征值的情况也就不奇怪了. 所以,或许将连通的圆盘看作一个整体,而不是单独考虑每个圆盘,会更有利于我们的讨论.

\begin{theorem}{Gershgorin 圆盘第二定理}{Strengthening of Gershgorin disks theorem}
    设 $T \in \mathcal{L}(V)$,并且 $v_1, \ldots, v_n$ 是 $V$ 的一组基,$D_i, i = 1, \ldots, n$ 是 $T$ 关于这组基的 Gershgorin 圆盘. 若 $\cup_{i = 1}^n D_i$ 是 $k$ 个不相交的连通区域 $R_1, \ldots, R_k$ 的并,并且 $R_r$ 是一个 $m_r$ 个 Gershgorin 圆盘的并,那么 $T$ 有 $m_r$ 个特征值落在 $R_r$ 中,$r = 1, \ldots, k$.
\end{theorem}

这一定理的证明需要用到特征值的连续性,这里不做过多展开. 借助于这一条定理,我们限制了每个连通区域内特征值的个数,从而使得我们可以更好地估计特征值的位置. 而自然地,我们也可以利用 Gershgorin 圆盘来刻画一些与特征值有关的性质.

\begin{corollary}{}{}
    \begin{enumerate}
        \item 若 $T$ 的 $n$ 个 Gershgorin 圆盘互不相交,那么 $T$ 可对角化;
        \item 若 $T$ 是实算子,且 $T$ 的 $n$ 个 Gershgorin 圆盘互不相交,那么 $T$ 的特征值都是实数.
    \end{enumerate}
\end{corollary}

\begin{proof}
    \begin{enumerate}
        \item 是平凡的,运用 $n$ 个不同的特征值是可对角化的充分条件即可;
        \item 考虑到实系数多项式的复根是成对出现并且共轭,以及实系数矩阵的 Gershgorin 圆盘的圆心都落在实数轴上即可.
    \end{enumerate}
\end{proof}

以上提到的 Gershgorin 圆盘都是使用的去心绝对行和作为半径,实际上我们也可以使用其他的方法来构造 Gershgorin 圆盘,比如使用去心绝对列和作为半径,即 \[
    C_i = \left\{z \in \mathbf{F} \mid \lvert z - a_{ii} \rvert \leqslant \sum_{j \neq i} \lvert a_{ji} \rvert\right\}, i = 1, \ldots, n.
\]
可以证明,这样构造的 Gershgorin 圆盘也满足 \nameref{thm:Gershgorin disks theorem}和 \nameref{thm:Strengthening of Gershgorin disks theorem}. 所以在估计的时候,我们可以根据具体情况选择适合的 Gershgorin 圆盘,或者将两种方法结合起来使用.

有些时候,我们还是觉得某个 Gershgorin 圆盘太大,无法给出特征值的精确估计. 这时,我们可以考虑将矩阵进行相似变换,使得新的矩阵对应的 Gershgorin 圆盘更小. 设 $D = \diag(p_1, p_2, \ldots, p_n), p_i > 0$ 是一个对角矩阵,那么有 \[
    D^{-1} A D = \begin{pmatrix}
        a_{11}                 & \frac{p_2}{p_1} a_{12} & \cdots & \frac{p_n}{p_1} a_{1n} \\
        \frac{p_1}{p_2} a_{21} & a_{22}                 & \cdots & \frac{p_n}{p_2} a_{2n} \\
        \vdots                 & \vdots                 & \ddots & \vdots                 \\
        \frac{p_1}{p_n} a_{n1} & \frac{p_2}{p_n} a_{n2} & \cdots & a_{nn}
    \end{pmatrix}.
\]

第 $j$ 行对应的 Gershgorin 圆盘的半径为 \[
    \frac{1}{p_j} \sum_{i \neq j} p_i \lvert a_{ij} \rvert,
\]

如果想要第 $j$ 行对应的 Gershgorin 圆盘更小,那么就需要让 $p_j$ 更大. 但是这样做的话,其他行对应的 Gershgorin 圆盘就会变大,所以我们需要在这些矛盾之间取得平衡.

\begin{summary}

    在本讲中我们首先从降低维数方便讨论的角度引入了对线性空间直和分解为更小的子空间的思想,引入了限制映射,最后要求限制映射是线性变换从而引出不变子空间的概念,并通过简单的例子验证、求解了不变子空间,更多的例子我们将会在学完若当标准形后见到.

    接下来我们从一维不变子空间入手,引入了特征值、特征向量以及特征子空间的概念,讨论了线性变换与矩阵在特征值、特征向量上的统一性,并分别研究了它们的性质. 在特征值中,我们首先说明了特征多项式的根就是特征值,特征值就是特征多项式的根,然后在\autoref{thm:特征多项式展开} 中讨论了特征多项式的展开式,特别是了解了特征值之和等于矩阵的迹,特征值之积等于矩阵的行列式的结论,并结合后面推导的有关于线性映射(矩阵)的倍数、幂次、逆、伴随、多项式的特征值的性质结论,在例题中体会了特征值性质的运用. 而对于特征向量和特征子空间,我们证明了不同特征值对应的特征向量是线性无关的,不同特征子空间之间是直和关系. 接下来我们结合了特征值和特征向量,证明了代数重数(特征多项式求解得到的特征值作为根的重数)大于等于几何重数(特征子空间的维数)的定理,也通过例题说明了即便所有特征子空间的直和等于全空间$V$,这也不表明$V$中所有向量都是特征向量,只有特征值唯一时才能做到这一点.

    然后我们讨论了实数域和复数域上的一些共性和差异性,首先是因为特征多项式在实数、复数域上解的情况不同导致的差别,但我们也通过``复化''的思想证明了即便实数域上可能没有特征值(即无一维不变子空间),但此时一定存在二维不变子空间. 最后我们还通过一个例子引入了实特征值和复特征值几何意义的不同(是单纯的长度放缩还是结合了旋转),尽管我们的讨论不完全严谨,但能提供一个良好的几何直观. 最后我们简单介绍了特征值的估计方法,即 Gershgorin 圆盘(第一、第二)定理,作为我们之前所学内容在特征值计算上的应用.

    从下一讲开始,我们将思考一个本讲中留待解决的问题:因为特征值的代数重数大于等于几何重数,这里的等于号并非总是能取到,因此所有特征子空间的直和不一定能够得到原空间$V$,因此我们要讨论代数重数和几何重数相等和不相等的时候如何对原空间进行分解使得我们可以获得较为简单的矩阵表示,继续我们简化矩阵表示的目标.

\end{summary}

\begin{exercise}
    \exquote[苏轼,《赤壁赋》]{盖将自其变者而观之,则天地曾不能以一瞬;自其不变者而观之,则物与我皆无尽也,而又何羡乎!}

    \begin{exgroup}
        \item 设$S,T\in \mathcal{L}(V)$满足$ST=TS$,证明:$\ker S$和$\im S$都在$T$下不变.
        \begin{answer}
            对于$\forall v\in\ker S,Sv=0\implies STv=TSv=0$,这表明$Tv\in \ker S$,$\ker S$在$T$下不变.
            对于$forall v \in \im S,v=Su\implies Tv=TSu=S(Tu)$,这表明$Tv \in \im S$,$\im S$在$T$下不变.
        \end{answer}
        \item 已知$\mathbf{R}^2$上线性变换$T$在基$e_1=(1,0),e_2=(0,1)$下的矩阵为$\begin{pmatrix}2 & 1 \\ 0 & 2\end{pmatrix}$. 证明:
        \begin{enumerate}
            \item 设$W_1$为由$e_1$张成的子空间,则$W_1$为$T$的不变子空间;

            \item $\mathbf{R}^2$不能表示为$T$的任何不变子空间$W_2$与$W_1$的直和.
        \end{enumerate}
        \begin{answer}
            \begin{enumerate}
                \item 对于$\forall v\in W_1$,有$v=ke_1,Tv=T(ke_1)=kT(e_1)=2ke_1\in W_1$,从而$W_1$为$T$的不变子空间.
                \item 容易看出$T$在自然基下的矩阵不可对角化,从而$\mathbf{R}^2$不能表示为$T$的任何不变子空间$W_2$与$W_1$的直和.
            \end{enumerate}
        \end{answer}

        \item 定义线性变换$T\in \mathcal{L}(\mathbf{F}^2)$为$T(x,y)=(y,0)$. 令$U=\{(x,0) \mid x\in\mathbf{F}\}$. 证明:
        \begin{enumerate}
            \item $U$在$T$下不变,且$T|_{U}$是$U$上的零线性变换;

            \item 不存在$\mathbf{F}^2$的在$T$下不变的子空间$W$使得$\mathbf{F}^2=U\oplus W$;

            \item $T/U$是$\mathbf{F}^2/U$上的0线性变换.
        \end{enumerate}
        \begin{answer}
            \begin{enumerate}
                \item 对于$\forall u=(x,0)\in U$,$Tu=T(x,0)=(0,0)\in U$,因此$U$在$T$下不变,且$T|_{U}$是$U$上的零线性变换.
                \item $T$在自然基下的矩阵为$\begin{pmatrix}
                    0 & 1 \\ 0 & 0
                \end{pmatrix}$,它不可对角化.
                \item 对于$\forall v+U=(x,y)+U\in \mathbf{F}^2/U$,$(T/U)(v+U)=Tv+U=(y,0)+U=0+U$,从而$T/U$是$\mathbf{F}^2/U$上的$0$线性变换.
            \end{enumerate}
        \end{answer}
        \item 设$V$是有限维的且$T\in \mathcal{L}(V)$,设$\lambda_1,\ldots,\lambda_m$是非零互异特征值,证明:
        \[ \dim E(\lambda_1,T)+\cdots+\dim E(\lambda_m,T)\leqslant\dim\im T. \]
        \begin{answer}
            考虑$U=E(\lambda_1,T)\oplus E(\lambda_2,T) \oplus \cdots \oplus E(\lambda_m,T)$,对于$\forall u\in U$,设$u=v_1+v_2+\cdots+v_m$,其中$v_i\in E(\lambda_i,T)$.则$v_i=T(\dfrac{v_i}{\lambda_i})$,从而$u=T(\displaystyle\sum_{i=1}^{m}\dfrac{v_i}{\lambda_i}) \in \im T$.因此$U$是$\im T$的子空间.进而有$\dim U=\dim E(\lambda_1,T)+\cdots+\dim E(\lambda_m,T) \leqslant \dim \im T$.
        \end{answer}
        \item 设$T\in \mathcal{L}(V)$且$\dim\im T=k$. 证明$T$至多有$k+1$个特征值.
        \begin{answer}
            设$T$的非零特征值为$\lambda_1,\ldots,\lambda_m$,则$m\leqslant \dim E(\lambda_1,T)+\cdots+\dim E(\lambda_m,T)\leqslant \dim\im T=k$.若$0$是$T$的非零特征值,则$T$至多有$k+1$个特征值.否则$T$至多有$k$个特征值.综上可知$T$至多有$k+1$个特征值.
        \end{answer}
        \item 设$\sigma$是线性空间$\mathbf{R}[x]_3$上的线性变换,它在基$1,x,x^2$下的矩阵为
        \[ A=\begin{pmatrix}
                1 & 2 & 2 \\ 2 & 1 & 2 \\ 2 & 2 & 1
            \end{pmatrix}\]
        求$\sigma$的特征值与特征子空间.
        \begin{answer}
            令$|\lambda E-A|=0$,解得$\lambda_1=5,\lambda_2=-1$.然后解$(5E-A)X=0$,解得$X=k(1,1,1)^\mathrm{T}$.解$(-E-A)X=0$,解得$X=k_1(1,-1,0)^\mathrm{T}+k_2(1,0,-1)^\mathrm{T}$.因此$\lambda_1=5$对应的特征子空间为$\spa(1+x+x^2)$,$\lambda_2=-1$对应的特征子空间为$\spa(1-x,1-x^2)$.
        \end{answer}
        \item 设$A,P$都是3阶方阵,$P$可逆,已知$A$的特征值$\lambda_1=1,\lambda_2=-1,\lambda_3=2$,$B=A^3-5A^2$,求$|B|$,$|A+5E|$,$|5E+P^{-1}AP|$.
        \begin{answer}
            矩阵的行列式等于其特征值之积,知$|A|=\lambda_1\lambda_2\lambda_3=-2$.
            进一步地,$A-5E$的特征值为$-4,-6,-3$,$A+5E$和$5E+P^{-1}AP$的特征值均为$6,4,7$.从而$|A-5E|=-72,|A+5E|=|5E+P^{-1}AP|=168$.
            $|B|=|A^2(A-5E)|=|A||A||A-5E|=-288$.
        \end{answer}
        \item 设$A=\begin{pmatrix}
                a & -1 & c \\ 5 & b & 3 \\ 1-c & 0 & -a
            \end{pmatrix}$,$|A|=-1$,$\alpha=(-1,-1,1)^\mathrm{T}$为$A^*$的特征向量,求$A^*$对应于$\alpha$的特征值及$a,b,c$和$A$对应于$\alpha$的特征值$\mu$.
        \begin{answer}
            设$A^*\alpha=\lambda \alpha$,则$AA^*\alpha=\lambda A\alpha$.从而$\lambda A\alpha=|A|\alpha=-\alpha\implies A\alpha=-\dfrac{1}{\lambda}\alpha$,故$\alpha$为$A$的特征向量.

            由$A\alpha=\begin{pmatrix}
                a & -1 & c \\ 5 & b & 3 \\ 1-c & 0 & -a
            \end{pmatrix}\cdot\begin{pmatrix}
                -1 \\ -1 \\ 1
            \end{pmatrix}=\begin{pmatrix}
                -a+1+c \\ -2-b \\ c-1-a
            \end{pmatrix}=-\dfrac{1}{\lambda}\alpha$可知:
            $(-a+1+c)+(c-1-a)=0$,从而$a=c$,代入可知$b=-3,-\dfrac{1}{\lambda}=-1,\lambda=1$.
            此外有$A=\begin{pmatrix}
                a & -1 & a \\ 5 & -3 & 3 \\ 1-a & 0 & -a
            \end{pmatrix}$,由$|A|=-1$解得$a=2$,因此$c=a=2$.
        \end{answer}

        \item 设$A,B\in \mathbf{M}_n(\mathbf{F})$,$AB=BA$,证明:若$X$是矩阵$A$属于特征值$\lambda_0$的特征向量,则$BX\in V_{\lambda_0}$(注:本题是解决很多$AB=BA$类问题的基础).
        \begin{answer}
            由于$AX=\lambda_0 X$,故$A(BX)=(AB)X=(BA)X=B(AX)=B(\lambda_0X)=\lambda_0(BX)$,从而$BX\in V_{\lambda_0}$.
        \end{answer}
    \end{exgroup}

    \begin{exgroup}
        \item 设$\sigma\in \mathcal{L}(V,W)$,定义$\widetilde{\sigma}:(V/(\ker \sigma))\to W$如下:
        \[\widetilde{\sigma}(v+\ker\sigma)=\sigma(v).\]
        \begin{enumerate}
            \item $\widetilde{\sigma}$是良定义的,且是$(V/(\ker \sigma))$到$W$上的线性映射;

            \item $\widetilde{\sigma}$是单射;

            \item $\im \widetilde{\sigma}=\im \sigma$;

            \item $V/(\ker \sigma)$同构于$\im \sigma$.
        \end{enumerate}
        \begin{answer}
            \begin{enumerate}
                \item 对于$v_1+\ker\sigma=v_2+\ker\sigma$,有$v_1=v_2+u$,其中$u\in\ker\sigma$.因此$\widetilde{\sigma}(v_1+\ker\sigma)=\sigma(v_1)=\sigma(v_1+u)=\sigma(v_2)=\widetilde{\sigma}(v_2+\ker\sigma)$,从而$\widetilde{\sigma}$是良定义的.

                对于$\forall v_1+\ker\sigma,v_2+\ker\sigma \in V/\ker\sigma$和$l,h\in \mathbf{F}$,都有$\widetilde{\sigma}(l(v_1+\ker\sigma)+h(v_2+\ker\sigma))=\widetilde{\sigma}(lv_1+hv_2+\ker\sigma)=\sigma(lv_1+hv_2)=l\sigma(v_1)+h\sigma(v_2)=l\widetilde{\sigma}(v_1+\ker\sigma)+h\widetilde{\sigma}(v_2+\ker\sigma)$,这表明$\widetilde{\sigma}$是$V/(\ker\sigma)$到$W$上的线性映射.
                \item 对于$v_1+\ker\sigma,v_2+\ker\sigma$,若有$\sigma(v_1)=\sigma(v_2)$,则$v_1-v_2\in \ker\sigma$,从而$v_1+\ker\sigma=v_2+\ker\sigma$.因此$\widetilde{\sigma}$为单射.
                \item 对于$\forall u\in \im\widetilde{\sigma}$,有$u \in \im\sigma$,故$\im \widetilde{\sigma} \subseteq \im\sigma$.

                对于$\forall v=\im(w) \in\im\sigma$,$v=\sigma(w)=\widetilde{\sigma}(w+\ker\sigma)\in\im\widetilde{\sigma}$,从而$\im\sigma\subseteq \im \widetilde{\sigma}$.综上有$\im \widetilde{\sigma}=\im\sigma$.
                \item 由$(2)(3)$可知$\widetilde{\sigma}$实际上给出了一个从$V/(\ker \sigma)$到$\im\sigma$的一一映射,从而$V/(\ker\sigma)\cong \im\sigma$.
            \end{enumerate}
        \end{answer}
        \item 设$T\in \mathcal{L}(V)$,证明:$T$是数乘变换(即$T=cI_V$,其中$c$为非零常数,$I_V$为$V$上的恒等映射)的充要条件是$V$的每一个一维子空间都是$T$的不变子空间.
        \begin{answer}
            必要性是显然的.下证充分性.
            设$v_1,v_2,\ldots,v_n$为$V$的一组基,则对于每个$1\leqslant i \leqslant n$,存在$c_i\in \mathbf{F}$,使得$T(v_i)=c_iv_i$.对于$\forall i\neq j$,$T(v_i+v_j)=c_iv_i+c_jv_j\in \spa(v_i+v_j)$,这表明$c_i=c_j$.从而所有的$c_i$均相等,$T$为数乘变换.
        \end{answer}

        \item 设$\sigma\in \mathcal{L}(V)$,证明:
        \begin{enumerate}
            \item $\sigma/(\im \sigma)=0$;

            \item $\sigma/(\ker \sigma)$是单射$\iff \ker \sigma\cap\im \sigma=\{0\}$.
        \end{enumerate}
        \begin{answer}
            \begin{enumerate}
                \item 对于$\forall v+\im\sigma$,$(\sigma/(\im\sigma))(v+\im\sigma)=\sigma(v)+\im\sigma=0+\im\sigma$,因此$\sigma/(\im\sigma)=0$.
                \item $\sigma/(\ker\sigma)$为单射$\iff \ker(\sigma/(\ker\sigma))=0+\ker\sigma\iff \ker\sigma\cap\im\sigma=\{0\}$.
            \end{enumerate}

        \end{answer}
        \item 设$V$是有限维的,$T\in \mathcal{L}(V)$且$U$在$T$下不变. 证明:$T/U$的每个特征值均为$T$的特征值.
        \begin{answer}
            对于$T/U$的特征值$\lambda$,设$(T/U)(v+U)=\lambda(v+U)=\lambda v+U$.则$(T/U)(v+U)=Tv+U=\lambda v+U$.
        \end{answer}
        \item 设$V$为$n$维复向量空间,$T\in \mathcal{L}(V)$,$T$在$V$的一组基$e_1,e_2,\ldots,e_n$下的矩阵为对角矩阵$\diag\{d_1,\ldots,d_n\}$,且$d_i\neq d_j\enspace(i\neq j)$.
        \begin{enumerate}
            \item 求$T$的所有一维不变子空间;

            \item 求$T$的所有不变子空间与个数.
        \end{enumerate}
        \begin{answer}
            \begin{enumerate}
                \item $T$的所有一维不变子空间为$\spa(e_i),i=1,2,\ldots,n$.
                \item $T$的所有不变子空间为所有$\spa(e_{i_1},e_{i_2},\ldots,e_{i_k})$,其中$(i_1,i_2,\ldots,i_k)\subset\{1,2,\ldots,n\}$.

                $T$的不变子空间个数为$2^n$.
            \end{enumerate}
        \end{answer}
        \item 给定$\mathbf{R}$上的2维线性空间$V$上的算子$T$,其在一组基$\alpha_1,\alpha_2$下的矩阵为
        \[\begin{pmatrix}
                0 & 1 \\ 1-a & 0
            \end{pmatrix}.\]
        求$T$的所有不变子空间.
        \begin{answer}
            令$|\lambda I-T|=0$,若$a<1$,则$\lambda_1=\sqrt{1-a},\lambda_2=-\sqrt{1-a}$,相应的特征向量为$(\alpha_1+\sqrt{1-a}\alpha_2),(\alpha_1-\sqrt{1-a}\alpha_2)$.从而$T$的所有不变子空间为$\{0\},\spa(\alpha_1+\sqrt{1-a}\alpha_2),\spa(\alpha_1-\sqrt{1-a}\alpha_2),V$.

            若$a=1$,则$\lambda=0$,相应的特征向量为$\alpha_1$,从而$T$的所有不变子空间为$\{0\},\spa(\alpha_1),V$.

            若$a>1$,则$T$在$\mathbf{R}$上无特征值,$T$的不变子空间只有$\{0\}$和$V$.
        \end{answer}
        \item 设$A,B$都是$n$阶矩阵,且$r(A)+r(B)<n$,证明:$A,B$有相同的特征值和特征向量.
        \begin{answer}
            由于$r(A)+r(B)<n$,故$A$和$B$都有特征值$0$.

            由于$\dim(N(A)\cap N(B))=\dim(N(A))+\dim(N(B))-\dim(N(A)+N(B))\geqslant\dim(N(A))+\dim(N(B))-n=n-r(A)+n-r(B)-n=n-(r(A)+r(B))\ge 0$,这表明$N(A)\cap N(B)\neq \{0\}$,从而存在$A$和$B$有相同的特征向量(对应于特征值$0$).
        \end{answer}

        \item 设$\lambda_1,\lambda_2,\ldots,\lambda_n$是矩阵$A=(a_{ij})_{n\times n}$的$n$个特征值,证明:$\lambda_1^2,\lambda_2^2,\ldots,\lambda_n^2$是$A^2$的$n$个特征值,且$\displaystyle\sum_{i=1}^{n}\lambda_i^2=\sum_{j=1}^{n}\displaystyle\sum_{k=1}^{n}a_{jk}a_{kj}$.
        \begin{answer}
            对于$\lambda_i,i=1,2,\ldots,n$,存在$u_i\in V,u_i\neq 0$,使得$Au_i=\lambda_iu_i$,从而$A^2u_i=A(\lambda_iu_i)=\lambda_iA(u_i)=\lambda_i^2u_i$,这表明$\lambda_i^2$为$A^2$的特征值,从而$\lambda_1^2,\lambda_2^2,\ldots,\lambda_n^2$为$A^2$的$n$个特征值.进而知$\displaystyle \sum_{i=1}^{n}\lambda_i^2=tr(A^2)=\sum_{j=1}^{n}\displaystyle\sum_{k=1}^{n}a_{jk}a_{kj}$.
        \end{answer}

        \item 设$A$为$n$阶矩阵,$X_1,X_2,X_3$为$n$元列向量,且$AX_1=kX_1\enspace(X_1\neq 0),AX_2=lX_1+kX_2,AX_3=lX_2+kX_3\enspace(l\neq 0)$. 证明:$X_1,X_2,X_3$线性无关.
        \begin{answer}
            设$a_1X_1+a_2X_2+a_3X_3=0$.在等式两边同时左乘$(A-kE)$可知:$a_2lX_1+a_3lX_2=0$.再左乘$(A-kE)$知$a_3l^2=0$.结合$l\neq 0$,有$a_3=0$.回代知$a_1=a_2=0$,从而$X_1,X_2,X_3$线性无关.
        \end{answer}

        \item \begin{enumerate}
            \item 使用 \nameref{thm:Gershgorin disks theorem} 证明:严格对角占优的矩阵是可逆的;
            \item 证明:矩阵 $A = \begin{pmatrix}
                          1 + \i & 0.2    & 0.2          \\
                          0.2    & 2 - \i & 0.2          \\
                          0.2    & 0.3    & -0.4 - 0.3\i
                      \end{pmatrix}
                  $ 是可逆的.
        \end{enumerate}
        \begin{answer}
            \begin{enumerate}
                \item 设$A=(a_{ij})_{n\times n}$是一个严格对角占优矩阵,则$A$的每个特征值都在某个Gershgorin圆盘$D_i=\{z\in \mathbf{F}||z-a_{ii}|\leqslant \displaystyle\sum_{j\neq i}|a_{ij}|\}$内.若$0$是$A$的特征值,则存在$i\in\{1,2,\ldots,n\}$,使得$|a_{ii}|<\displaystyle\sum_{j\neq i}|a_{ij}|$,这与$A$是严格对角占优矩阵矛盾.从而$0$不是$A$的特征值,$A$为可逆矩阵.
                \item 假设$A$不可逆,则$A$有特征值$0$.显然$0\notin D_1,D_2$,故$0\in D_3$.注意到$0$实际上在$D_3$的边界上,回顾Gershgorin圆盘第一定理的证明,要使得等号成立则应有$c_1=c_2=c_3$,从而$(v_1+v_2+v_3)$是$A$对应于特征值$0$的特征向量.而$(v_1+v_2+v_3)A\neq 0$,矛盾.故假设不成立,$A$为可逆矩阵.
            \end{enumerate}
        \end{answer}
    \end{exgroup}

    \begin{exgroup}
        \item 设$A,B\in \mathbf{M}_n(\mathbf{C})$,$B$的特征多项式$f(\lambda)=|\lambda E-B|$. 证明:$f(A)$可逆的充要条件是$B$的任一特征值都不是$A$的特征值.
        \begin{answer}
            $f(\lambda)=|\lambda E-B|=(\lambda-\lambda_1)^{\dim E(\lambda_1,B)}\cdots(\lambda-\lambda_m)^{\dim E(\lambda_m,B)}$,故$f(A)$可逆$\iff (A-\lambda_1E)^{\dim E(\lambda_1,B)}\cdots(A-\lambda_mE)^{\dim E(\lambda_m,B)}$可逆$\iff \mid(A-\lambda_1E)^{\dim E(\lambda_1,B)}\cdots(A-\lambda_mE)^{\dim E(\lambda_m,B)}\mid \neq 0 \iff \mid(A-\lambda_1E)\mid^{\dim E(\lambda_1,B)}\cdots\mid(A-\lambda_mE)\mid^{\dim E(\lambda_m,B)}\neq 0 \iff \forall i=1,2,\ldots,m,|A-\lambda_mE|\neq 0 \iff B$的任一特征值都不是$A$的特征值.
        \end{answer}

        \item 证明:若$AB=BA$,则$A$和$B$至少有一个共同的特征向量.
        \begin{answer}
            命题等价于$\sigma,\tau \in \mathcal{L}(V),\sigma\tau=\tau\sigma$,则$\sigma$和$\tau$至少有一个公共的特征向量.

            考虑$\sigma$的不变子空间$E(\lambda,\sigma)$,对于$v\in E(\lambda,\sigma)$,有$\sigma v=\lambda v \implies \sigma(\tau v)=\tau\sigma v=\lambda (\tau v)$,这表明$E(\lambda,\sigma)$是$\tau$的不变子空间.考虑限制映射$\tau|_E(\lambda,\sigma)$,它一定有特征值和相应的特征向量$u\in E(\lambda,\sigma),u \neq 0$,从而$u$为$\sigma$和$\tau$的公共特征向量.
        \end{answer}
    \end{exgroup}
\end{exercise}
