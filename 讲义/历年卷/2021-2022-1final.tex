\phantomsection
\section*{2021-2022学年线性代数I(H)期末}
\addcontentsline{toc}{section}{2021-2022学年线性代数I(H)期末}

\begin{center}
    任课老师:统一命卷\hspace{4em} 考试时长:120分钟
\end{center}

\begin{enumerate}
	\item[一、](12分)定义实数域上的线性空间$\mathbf{R}^n$到自身的映射$T$如下:
	\[\forall X=(x_1,x_2,\ldots,x_n)\in\mathbf{R}^n,\enspace T(X)=(x_1-x_2,x_2-x_3,\ldots,x_{n-1}-x_n,x_n-x_1).\]
    \begin{enumerate}[label=(\arabic*)]
        \item 验证$T\in\mathcal{L}(\mathbf{R}^n)$;

        \item 求$T$的像空间,和$T$核空间的维数.
    \end{enumerate}
	\item[二、](12分)设
	\[A=\begin{pmatrix}
        1 & 1 & -1 & -1 \\ 2 & 2 & 1 & 0 \\ 3 & 3 & 0 & -1 \\ 1 & 1 & 2 & 0
    \end{pmatrix},\enspace b=\begin{pmatrix}
        0 \\ 1 \\ 1 \\ 1
    \end{pmatrix},\enspace X=\begin{pmatrix}
        x_1 \\ x_2 \\ x_3 \\ x_4
    \end{pmatrix}.\]
    求线性方程组$AX=b$的一般解.
	\item[三、](12分)设三元二次齐次实多项式如下:
	\[f(x,y,z)=x^2+2xy+y^2-6yz-4xz.\]
    \begin{enumerate}[label=(\arabic*)]
        \item 求实对称矩阵$A$,使得$f(x,y,z)=\begin{pmatrix}
            x & y & z
        \end{pmatrix}A\begin{pmatrix}
            x & y & z
        \end{pmatrix}^\mathrm{T}$;

        \item 求一个与$A$合同的对角矩阵;

        \item 求$f(x,y,z)$的正惯性指数和负惯性指数.
    \end{enumerate}
	\item[四、](12分)设$V=\mathbf{R}^{4\times 1}$,$W=\mathbf{R}^{3\times 1}$,定义映射$T:V\to W$如下:
	\[T(X)=AX=\begin{pmatrix}
        1 & -1 & 0 & 1 \\ 1 & 1 & 2 & 3 \\ 2 & 2 & 3 & 4
    \end{pmatrix}\begin{pmatrix}
        x_1 \\ x_2 \\ x_3 \\ x_4
    \end{pmatrix},\enspace\forall X=\begin{pmatrix}
        x_1 \\ x_2 \\ x_3 \\ x_4
    \end{pmatrix}\in V.\]
    \begin{enumerate}[label=(\arabic*)]
        \item 证明$T$的秩为3;

        \item 求$V$和$\im T$的基$\{\varepsilon_1,\varepsilon_2,\varepsilon_3,\varepsilon_4\}$和$\{\eta_1,\eta_2\}$,使得
        \[T(\varepsilon_1)=\eta_1,T(\varepsilon_2)=\eta_2,T(\varepsilon_3)=T(\varepsilon_4)=(0,0,0)^\mathrm{T}.\]
    \end{enumerate}
	\item[五、](12分)设 $V=\mathbb{R}^{2 \times 2}$ 是按矩阵加法和数乘构成的实数域上的线性空间.
    \begin{enumerate}[label=(\arabic*)]
        \item 验证下列向量组构成 $V$ 的一组基:
    \[B=\left\{\begin{bmatrix}
    1 & 0 \\ 0 & 0 \end{bmatrix},\begin{bmatrix}
    0 & 1 \\ 0 & 0 \end{bmatrix},\begin{bmatrix}
    1 & 1 \\ 1 & 0 \end{bmatrix},\begin{bmatrix}
    1 & 1 \\ 1 & 1 \end{bmatrix}\right\};\]

        \item 在 $V$ 上定义运算
        \[\sigma\left(\left(a_{ij}\right)_{2 \times 2},\left(b_{ij}\right)_{2 \times 2}\right)=a_{11} b_{11}+a_{12} b_{12}+a_{21} b_{21}+a_{22} b_{22}.\]

        验证 $\sigma$ 是 $V$ 上一个内积,使得 $V$ 成为一个欧氏空间;

        \item 将 Schmidt 正交化过程用于 $B$ 求出 $V$ 的一组单位正交基.
    \end{enumerate}
	\item[六、](8分)求矩阵
	\[A=\begin{bmatrix}
    0 & -1 & 1 \\
    -1 & 0 & 1 \\
    1 & 1 & 0
    \end{bmatrix}\]
    的所有特征值,对应的特征子空间,以及与 $A$ 相似的一个对角矩阵.
	\item[七、](16分)设 $V=\mathbb{R}^3$ 是具有自然内积的欧氏空间,$T \in L(V)$. 设
	\[\alpha_1=(1,2,0), \;\alpha_2=(0,1,2), \;\alpha_3=(2,0,1);\]
    \[T\left(\alpha_1\right)=-(1,0,2), \;T\left(\alpha_2\right)=-(2,1,0), \;T\left(\alpha_3\right)=-(0,2,1).\]
    \begin{enumerate}[label=(\arabic*)]
        \item 求 $T$ 关于 $V$ 的自然基的矩阵;

        \item 证明 $T$ 是一个正交变换;

        \item 证明 $T$ 是一个镜面反射变换.(存在 $V$ 的单位正交基 $\{\eta,\; \beta,\; \gamma\}$ 使得 $T(\eta)=-\eta,\; T(\beta)=\beta,\; T(\gamma)=\gamma$,或等价地,存在单位向量 $\eta$ 使得 $T(\alpha)=\alpha-2(\alpha,\; \eta) \eta,\; \forall \alpha \in V$)
    \end{enumerate}
	\item[八、](16分)判断下列命题的真伪,若它是真命题,请给出简单的证明;若它是伪命题,给出理由或举反例将它否定.
    \begin{enumerate}[label=(\arabic*)]
        \item 设 $A$ 是实数域上 $m \times n$ 阶矩阵,则矩阵秩 $r\left(A^T A\right)=r(A)$;

        \item 设 $A$ 是复数域上 $m \times n$ 阶矩阵,则矩阵秩 $r\left(A^T A\right)=r(A)$;

        \item 设 $V, W$ 是数域 $F$ 上的线性空间,则 $V \cup W$ 是线性空间;

        \item 实矩阵的下列性质有其二必有其三:
        \begin{enumerate}
            \item $A^T=A$;

            \item $A^T A=E$(单位矩阵);

            \item $A^2=E$.
        \end{enumerate}
    \end{enumerate}
\end{enumerate}

\clearpage
