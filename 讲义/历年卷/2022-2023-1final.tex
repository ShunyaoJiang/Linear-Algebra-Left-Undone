\phantomsection
\section*{2022-2023学年线性代数I(H)期末}
\addcontentsline{toc}{section}{2022-2023学年线性代数I(H)期末}

\begin{center}
    任课老师:统一命卷\hspace{4em} 考试时长:120分钟
\end{center}

\begin{enumerate}
	\item[一、](10分)求线性方程组

	\[\left\{\begin{matrix}
	  x_1+kx_2+x_3= 1\\
	  x_1-x_2+x_3= 1 \\
	  kx_1+x_2+2x_3= 1 \\
	\end{matrix}\right.\]

	在 $k$ 为多少时有解,并求出一般解.
	\item[二、](10分)给定二次型 $f(x_1,x_2,x_3)=x_1^2-x_3^2+2x_1x_2+2x_2x_3$,请将其化为标准型,并求出此时的线性变换矩阵,以及该二次型的正、负惯性指数.
	\item[三、] (10分)已知三阶矩阵 $A$ 的伴随矩阵 $A^*=\begin{pmatrix}
		3 & -1 &- 1\\
		-1 & 3 & -1\\
		-1 & -1 & 3
	  \end{pmatrix} $,求 $A$.
	\item[四、](10分)设 $A\in \mathbf R^{p\times m},B\in \mathbf R^{m\times n},r(A)=r,r(B)=s,r(AB)=t$. 令 $V=\{X\in \mathbf R^n\mid ABX=0\},W=\{Y\in \mathbf R^m\mid Y=BX,X\in V\}$.
      \begin{enumerate}[label=(\arabic*)]
        \item 证明 $V$ 是 $\mathbf R^n$ 上的子空间,$W$ 是 $\mathbf R^m$ 上的子空间.
        \item 求 $\dim V,\dim W$.
      \end{enumerate}

	\item[五、](10分)设三阶矩阵 $A$,满足 $|A-E|=|A-2E|=|A+E|=0$.
      \begin{enumerate}[label=(\arabic*)]
        \item 求 $A$ 的所有特征值.
        \item 求 $|A+3E|$.
      \end{enumerate}

	\item[六、](15分)
      \begin{enumerate}[label=(\arabic*)]
        \item 设 $A$ 为 $n$ 阶矩阵,满足 $r(A)=r$,证明:存在可逆的矩阵 $P$,使得 $P^{-1}AP$ 的后 $n-r$ 列均为 0.
        \item 设 $A$ 为 $n$ 阶矩阵,满足 $r(A)=1$,$A$ 主对角线上元素之和为 1,证明:$A^2=A$.
      \end{enumerate}

	\item[七、](15分)定义 $\mathbf R_3[x]=\{a_2x^2+a_1x+a_0\mid a_0,a_1,a_2\in \mathbf R\}$ . 设 $\mathbf R_3[x]$ 对 $\mathbf R^{2\times 2}$ 的映射 $\sigma$ 满足:

	\[\sigma(p(x))=\begin{pmatrix}
	    p(1)-p(2)&0\\
	    0 & p(0)\end{pmatrix}\]

        \begin{enumerate}[label=(\arabic*)]
            \item 证明:$\sigma$ 为线性映射.
            \item 试分别写出 $\mathbf R_3[x],\mathbf R^{2\times 2}$ 上的两组基 $B_1,B_2$,并求出 $\sigma$ 关于这两组基的矩阵.
            \item 求 $\text{Im}\sigma,\ker\sigma$.
            \item 分别给出 $\mathbf R_3[x]$ 的一个与 $\text{Im}\sigma$ 同构的子空间,和 $\mathbf R^{2\times 2}$ 的一个与 $\text{Ker}\sigma$ 同构的子空间.
        \end{enumerate}

	\item[八、](20分)判断下列命题的真伪,若它是真命题,请给出简单的证明;若它是伪命题,给出理由或举反例将它否定.
    \begin{enumerate}[label=(\arabic*)]
        \item 若 $\alpha_1,\alpha_2,\dots,\alpha_n,\beta$ 的秩大于 $\alpha_1,\alpha_2,\dots,\alpha_n$ 的秩,则 $\alpha_1,\alpha_2,\dots,\alpha_n,\beta$ 线性无关;
        \item 设 $U,V,W$ 为 $V_0$ 关于数域 $F$ 的线性空间,若 $U+V=U+W$,则 $V=W$;
        \item 任意不为 0 矩阵的二阶矩阵可以表示为若干初等矩阵的乘积;
        \item 若 $A,B$ 相似或者相合,则 $A,B$ 相抵.
    \end{enumerate}
\end{enumerate}

\clearpage
